\documentclass{article}
\usepackage{enumerate}
\usepackage{amsmath}
\usepackage{amssymb}
\usepackage{graphicx}
\usepackage{subfigure}
\usepackage{geometry}
\usepackage{caption}
\usepackage{indentfirst}
\usepackage{multirow}
\usepackage{algorithm}  
\usepackage{algorithmicx}  
\usepackage{algpseudocode}
\geometry{left=3.0cm,right=3.0cm,top=3.0cm,bottom=4.0cm}
\renewcommand{\thesection}{Ex. \arabic{section} ---}
\renewcommand{\mod}{{\rm\ mod\ }}
\title{VE475 Homework 10}
\author{Liu Yihao 515370910207}
\date{}

\begin{document}
\maketitle

\section{Group structure on an elliptic curve}
\begin{align*}
x_3^3+bx_3+c&=(m^2-x_1-x_2)^3+b(m^2-x_1-x_2)+c\\
&=m^6-3m^4x_1-3m^4x_2+3m^2x_1^2+6m^2x_1x_2+3m^2x_2^2+bm^2\\
&\quad-x_1^3-3x_1^2x_2-3x_1x_2^2-bx_1-x_2^3-bx_2+c\\
y_3^2&=m^2(2x_1+x_2-m^2)^2-2m(2x_1+x_2-m^2)y_1+y_1^2\\
&=m^6-4m^4x_1-2m^4x_2+2m^3y_1+4m^2x_1x_2+m^2x_2^2-4mx_1y_1-2mx_2y_1
+y_1^2\\
x_3^3+bx_3+c-y_3^2&=m^4x_1-m^4x_2-2m^3y_1-m^2x_1^2+2m^2x_1x_2+2m^2x_2^2+bm^2\\
&\quad+4mx_1y_1+2mx_2y_1-x_1^3-3x_1^2x_2-3x_1x_2^2-bx_1-x_2^3-bx-2-y_1^2+c
\end{align*}

When $P_1\neq P_2$, $m=\dfrac{y_2-y_1}{x_2-x_1}$
\begin{align*}
x_3^3+bx_3+c-y_3^2&=
-\frac{1}{(x_1-x_2)^3}(x_1^6 - 3x_1^4x_2^2 + bx_1^4 - 2bx_1^3x_2 - 2x_1^3y_1^2 + 2x_1^3y_1y_2 + x_1^3y_2^2 - cx_1^3 \\
&\quad + 3x_1^2x_2^4 - 3x_1^2x_2y_2^2 + 3cx_1^2x_2 + 2bx_1x_2^3 + 3x_1x_2^2y_1^2 - 3cx_1x_2^2 - bx_1y_1^2 + 2bx_1y_1y_2 \\
&\quad  - bx_1y_2^2- x_2^6 - bx_2^4 - x_2^3y_1^2 - 2x_2^3y_1y_2 + 2x_2^3y_2^2 + cx_2^3 + bx_2y_1^2 - 2bx_2y_1y_2 + bx_2y_2^2 \\
&\quad + y_1^4 - 2y_1^3y_2 + 2y_1y_2^3 - y_2^4)
\end{align*}

Since $y_1^2=x_1^3+bx+c$, $y_2^2=x_2^3+bx+c$, we can get
\begin{align*}
x_3^3+bx_3+c-y_3^2&=-\frac{1}{(x_1-x_2)^3}[
x_1^3(x_2^3 + bx_2 + c) - x_2^3(x_1^3 + bx_1 + c) - 2x_1^3(x_1^3 + bx_1 + c)\\
&\quad + 2x_2^3(x_2^3 + bx_2 + c) + 3x_1^2x_2^4 - 3x_1^4x_2^2 + (x_1^3 + bx_1 + c)^2 - (x_2^3 + bx_2 + c)^2\\
&\quad + bx_1^4 - bx_2^4 - cx_1^3 + cx_2^3 + x_1^6 - x_2^6 - bx_1(x_1^3 + bx_1 + c) + bx_2(x_1^3 + bx_1 + c)\\
&\quad - bx_1(x_2^3 + bx_2 + c) + bx_2(x_2^3 + bx_2 + c) + 2bx_1x_2^3 - 2bx_1^3x_2 - 3cx_1x_2^2 + 3cx_1^2x_2\\
&\quad + 2y_1y_2(x_2^3 + bx_2 + c) + 2x_1^3y_1y_2 - 2x_2^3y_1y_2 + 3x_1x_2^2(x_1^3 + bx_1 + c)\\
&\quad - 3x_1^2x_2(x_2^3 + bx_2 + c) - y_1y_2(2x_1^3 + 2bx_1 + 2c) + 2bx_1y_1y_2 - 2bx_2y_1y_2]\\
&=0
\end{align*}
So $$y_3^2=x^3_3+bx_3+c$$ \\

When $P_1=P_2$, $x_1=x_2$, $y_1=y2$, $m=\dfrac{3x_1^2+b}{2y_1}$
$$x_3^3+bx_3+c-y_3^2=x_1^3 + bx_1 - y_1^2 + c$$

Since $y_1^2=x_1^3+bx+c$, we can get

$$y_3^2=x_3^3+bx_3+c$$

So the addition law over $E$ is proved.\\

Then we need to proved the commutative law, which means for $P_1,P_2\in E$ $P_1+P_2=P_2+P_1$. Suppose $P_1+P_2=(x,y)$, $P_2+P_1=(x',y')$, 
first, when $P_1=P_2$, it is obviously true. Otherwise, we know $m=m'=\dfrac{y_2-y_1}{x_2-x_1}$.
$$x=x'=m^2-x_1-x_2$$
$$y=m(x_1-x)-y_1=\frac{(x_1-x)(y_2-y_1)-(x_2-x_1)y_1}{x_2-x_1}=\frac{x_1y_2-x_2y_1-x(y_2-y_1)}{x_2-x_1}$$
$$y'=m(x_2-x)-y_2=\frac{(x_2-x)(y_2-y_1)-(x_2-x_1)y_2}{x_2-x_1}=\frac{x_1y_2-x_2y_1-x(y_2-y_1)}{x_2-x_1}$$
$$y=y'$$

So the commutative law is proved. \\

At last we need to prove the associative law, which means for $P_1,P_2,P_3\in E$, $(P_1+P_2)+P_3=P_1+(P_2+P_3)$.

Suppose we have 9 points: $\mathcal{O},P_1,P_2,P_3,P_1+P_2,P_2+P_3,-(P_1+P_2),-(P_2+P_3),-(P_1+P_2+P_3)$. Connect them with 6 lines, the sum of the three values on any of them is zero. The central point $-(P_1+P_2+P_3)$ lies on the line through $P_1$ and $P_2+P_3$, and also lies on the line through $P_1+P_2$ and $P_3$. Then we can conclude that 
$(P_1+P_2)+P_3=P_1+(P_2+P_3)$.

So the associative law is proved.


\section{Number of points on an elliptic curve}
\begin{enumerate}
\item
$$m_2\equiv\frac{3x_1^2+3}{2y_1}\equiv9\mod11$$
$$x_2\equiv m_2^2-2x_1\equiv10\mod11$$
$$y_2\equiv m_2(x_1-x_2)-y_1\equiv6\mod11$$
$$[2]P=(10,6)$$

$$m_4\equiv\frac{3x_2^2+3}{2y_2}\equiv6\mod11$$
$$x_4\equiv m_4^2-2x_2\equiv5\mod11$$
$$y_4\equiv m_4(x_4-x_4)-y_2\equiv2\mod11$$
$$[4]P=(5,2)$$

$$m_5\equiv\frac{y_4-y_1}{x_4-x_1}\equiv6\mod11$$
$$x_5\equiv m_5^2-x_4-x_1\equiv1\mod11$$
$$y_5\equiv m_5(x_4-x_5)-y_4\equiv0\mod11$$
$$[5]P=(1,0)$$

Since $y_5=0$,
$$[10]P=(0,0)$$

\item
There are 10 points.

\item \ 
\begin{center}
\begin{tabular}{c|c|c|c}
$x\mod11$ & $y^2\mod11$ & $y\mod11$ & Points on $E$\\\hline
0  & 7  & /\\
1  & 0  & /\\
2  & 10 & 0 & (1,0)\\
3  & 10 & /\\
4  & 6  & / \\
5  & 4  & 2 or 9 & (5,2) or (5,9) \\
6  & 10 & /\\
7  & 8  & /\\
8  & 4  & 2 or 9 & (8,2) or (8,9) \\
9  & 4  & 2 or 9 & (9,2) or (9,9) \\
10 & 3  & 5 or 6 & (10,5) or (10,6) \\
\end{tabular}
\end{center}

The elliptic curve $E$ has 10 points: 9 calculated from the
equation plus the point at the infinity $\mathcal{O}$.

\end{enumerate}

\section{ECDSA}
In the Elliptic Curve Digital Signature Algorithm, we need a curve $E$, Point $G\in E$ and the order $n$ of $G$ which means $[n]G=\mathcal{O}$. We also need a cryptographic hash function $h$.\\

Alice creates a key pair, consisting of a private key integer $d_{A}$, randomly selected in the interval $[1,n-1]$, and a public key curve point $Q_A=[d_A]G$.\\

When Alice wants to sign a message $m$, the procedure is:
\begin{enumerate}
\item Calculate $e=h(m)$.
\item Let $z$ be $L_n$ leftmost bits of $e$, where $L_n$ is the  bit length of the group order $n$.
\item Generate a random integer $k$ in $[1,n-1]$.
\item Calculate $P:(x_1,y_1)=[k]G$.
\item Calculate $r\equiv x_1\mod n$. If $r=0$, retry from step 3.
\item Calculate $s\equiv k^{-1}(z+rd_A)\mod n$. If $s=0$, retry from step 3.
\item The signature is the pair $(r,s)$.
\end{enumerate}

When Bob wants to authenticate Alice's signature, he must have a copy of her public-key curve point $Q_A$. First he can verify $Q_A$ is a valid curve point as follows:

\begin{enumerate}
\item Check that $Q_{A}$ is not equal to the identity element $\mathcal{O}$.
\item Check that $Q_{A}$ lies on the curve.
\item Check that $[n]Q_{A}=\mathcal{O}$.
\end{enumerate}

After that, Bob follows these steps:

\begin{enumerate}
\item Verify that $r$ and $s$ are integers in $[1,n-1]$. If not, the signature is invalid.
\item Calculate $e=h(m)$.
\item Let $z$ be $L_n$ leftmost bits of $e$, where $L_n$ is the  bit length of the group order $n$.
\item Calculate $w\equiv s^{-1}\mod n$
\item Calculate $u_1\equiv zw\mod n$ and $u_2\equiv rw\mod n$.
\item Calculate the curve point $P:(x_1,y_1)=[u_1]G+[u_2]Q_A$. If $P=\mathcal{O}$, the signature is invalid.
\item The signature is valid if $r\equiv x_1\mod n$, invalid otherwise.
\end{enumerate}

Now we should validate the correctness of the algorithm. Suppose we have $P:(x_1,y_1)=[u_1]G+[u_2]Q_A$ in step 6 of authentication. 
\begin{align*}
P&=[u_1]G+[u_2]Q_A\\
&=[u_1+u_2d_A]G\\
&=[zs^{-1}+rd_As^{-1}]G\\
&=[(z+rd_A)s^{-1}]G\\
&=[(z+rd_A)(k^{-1}(z+rd_A))^{-1}]G\\
&=[k]G
\end{align*}

The benefits of ECDSA is that we can get the same level of security as RSA but with smaller keys. Smaller keys have faster algorithms for generating signatures because the math involves smaller numbers. Smaller public keys mean smaller certificates and less data to pass around to establish a TLS connection. This means quicker connections and faster loading times on websites.

\section{BB84}
In the BB84 scheme, Alice wishes to send a private key to Bob. She begins with two strings of bits, $a$ and $b$, each $n$ bits long. She then encodes these two strings as a string of $n$ qubits:
$$|\psi\rangle=\bigotimes_{i=1}^n|\psi_{a_ib_i}\rangle$$
$$|\psi_{00}\rangle=|0\rangle$$
$$|\psi_{10}\rangle=|1\rangle$$
$$|\psi_{01}\rangle=\frac{1}{\sqrt{2}}|0\rangle+\frac{1}{\sqrt{2}}|1\rangle$$
$$|\psi_{11}\rangle=\frac{1}{\sqrt{2}}|0\rangle-\frac{1}{\sqrt{2}}|1\rangle$$

Alice sends $|\psi \rangle$  over a public and authenticated quantum channel $\mathcal {E}$ to Bob. Since only Alice knows $b$, it makes it virtually impossible for either Bob or Eve to distinguish the states of the qubits. Also, after Bob has received the qubits, we know that Eve cannot be in possession of a copy of the qubits sent to Bob, by the no-cloning theorem, unless she has made measurements. Her measurements, however, risk disturbing a particular qubit with probability 1/2 if she guesses the wrong basis.

Bob proceeds to generate a string of random bits $b'$ of the same length as $b$ and then measures the string he has received from Alice, $a'$. At this point, Bob announces publicly that he has received Alice's transmission. Alice then knows she can now safely announce $b$. Bob communicates over a public channel with Alice to determine which $b_{i}$ and $b'_{i}$ are not equal. Both Alice and Bob now discard the qubits in $a$ and $a'$ where $b$ and $b'$ do not match.

From the remaining $k$ bits where both Alice and Bob measured in the same basis, Alice randomly chooses $k/2$ bits and discloses her choices over the public channel. Both Alice and Bob announce these bits publicly and run a check to see whether more than a certain number of them agree. If this check passes, Alice and Bob proceed to use information reconciliation and privacy amplification techniques to create some number of shared secret keys. Otherwise, they cancel and start over.

\section{Quantum key distribution}
\begin{enumerate}
\item
Alice and Bob use the quantum channel to produce and distribute a key, and they use the classic channel to send message encrypted by that key.
\item
When Eve tries to observe the information on the quantum channel, according to quantum indeterminacy, measuring an unknown quantum state changes that state in some way. When this happens, Alice and Bob can easily detect the interaction, so that they can begin with a new key which is not disturbed.
\end{enumerate}

\section{Simple questions}
\begin{enumerate}
\item
\begin{align*}
U_1\otimes V_1&=
\begin{pmatrix}
u_{1,1,1}V_1 & u_{1,1,2}V_1 & \cdots & u_{1,1,n}V_1 \\
u_{1,2,1}V_1 & u_{1,2,2}V_1 & \cdots & u_{1,2,n}V_1 \\
\vdots & \vdots & \ddots & \vdots \\
u_{1,n,1}V_1 & u_{1,n,2}V_1 & \cdots & u_{1,n,n}V_1 \\
\end{pmatrix}
\\&=
\begin{pmatrix}
u_{1,1,1}v_{1,1,1} & \cdots & u_{1,1,1}v_{1,1,n} & \cdots & \cdots & u_{1,1,n}v_{1,1,1} & \cdots & u_{1,1,n}v_{1,1,n} \\
\vdots & \ddots & \vdots & \ddots & \ddots & \vdots & \ddots & \vdots \\
u_{1,1,1}v_{1,n,1} & \cdots & u_{1,1,1}v_{1,n,n} & \cdots & \cdots & u_{1,1,n}v_{1,n,1} & \cdots & u_{1,1,n}v_{1,n,n} \\
\vdots & \vdots & \vdots & \vdots & \vdots & \vdots & \vdots & \vdots \\
\vdots & \vdots & \vdots & \vdots & \vdots & \vdots & \vdots & \vdots \\
u_{1,1,n}v_{1,1,1} & \cdots & u_{1,1,n}v_{1,1,n} & \cdots & \cdots & u_{1,n,n}v_{1,1,1} & \cdots & u_{1,n,n}v_{1,1,n} \\
\vdots & \ddots & \vdots & \ddots & \ddots & \vdots & \ddots & \vdots \\
u_{1,1,n}v_{1,n,1} & \cdots & u_{1,1,n}v_{1,n,n} & \cdots & \cdots & u_{1,n,n}v_{1,n,1} & \cdots & u_{1,n,n}v_{1,n,n} \\
\end{pmatrix}
\end{align*}
Let $i_1=\lceil i/n \rceil$, $j_1=\lceil j/n \rceil$, $i_2=(i-1\mod n)+1$, $j_2=(j-1\mod n)+1$
$$W_1=U_1\otimes V_1,W_2=U_2\otimes V_2,X=W_1\cdot W_2$$
$$w_{1,i,j}= u_{1,i_1,j_1}v_{1,i_2,j_2} \quad (i,j\in[1,n^2])$$
$$w_{2,i,j}= u_{2,i_1,j_1}v_{2,i_2,j_2} \quad (i,j\in[1,n^2])$$
Similarly, let $k_1=\lceil k/n \rceil$, $k_2=(k-1\mod n)+1$
$$x_{i,j}=\sum_{k=1}^{n^2}w_{1,k,j}w_{2,i,k}=\sum_{k=1}^{n^2}u_{1,k_1,j_1}u_{2,i_1,k_1}v_{1,k_2,j_2}v_{1,i_2,k_2} \quad (i,j\in[1,n^2])$$

$$W_3=U_1\cdot U_2, W_4=V_1\cdot V_2, Y=W_3\otimes W_4$$
$$w_{3,i,j}\sum_{k=1}^{n}u_{1,k,j}u_{2,i,k} \quad (i,j\in[1,n])$$
$$w_{4,i,j}\sum_{k=1}^{n}v_{1,k,j}v_{2,i,k} \quad (i,j\in[1,n])$$
$$y_{i,j}=\sum_{k=1}^{n}u_{1,k,j_1}u_{2,i_1,k}\sum_{k=1}^{n}v_{1,k,j_2}v_{2,i_2,k}=\sum_{k=1}^{n^2}u_{1,k_1,j_1}u_{2,i_1,k_1}v_{1,k_2,j_2}v_{1,i_2,k_2}  \quad (i,j\in[1,n^2])$$

So $$(U_1\otimes V_1)\cdot(U_2\otimes V_2)=(U_1U_2)\otimes(V_1V_2)$$

\item
If we have two vector spaces $V,W$, while $V$ have a basis $e_1,e_2,\cdots,e_m$ and $W$ have a basis $f_1,f_2,\cdots,f_n$, then $U=V\otimes W$ is a vector space with basis $e_i\otimes f_j$. For any vector $v=\sum_{i=0}^mv_ie_i\in V$ and $w=\sum_{j=0}^mw_jf_j\in V$, we can get $u=v\otimes w=\sum_{i=0}^{m}\sum_{j=0}^{n}v_iw_j (e_i\otimes f_j)$ so that $u\in U$. So the operator $\otimes$ is bilinear.

\end{enumerate}

\end{document}
