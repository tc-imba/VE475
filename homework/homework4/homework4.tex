\documentclass{article}
\usepackage{enumerate}
\usepackage{amsmath}
\usepackage{amssymb}
\usepackage{graphicx}
\usepackage{subfigure}
\usepackage{geometry}
\usepackage{caption}
\usepackage{indentfirst}
\usepackage{multirow}
\geometry{left=3.0cm,right=3.0cm,top=3.0cm,bottom=4.0cm}
\renewcommand{\thesection}{Ex. \arabic{section} ---}
\renewcommand{\mod}{{\rm\ mod\ }}
\title{VE475 Homework 4}
\author{Liu Yihao 515370910207}
\date{}

\begin{document}
\maketitle

\section{Euler's totient}
\begin{enumerate}
\item 
Suppose $$\varphi(p^k)=p^{k-1}(p-1)=p^k-p^{k-1}$$ which means, there are $p^{k-1}$ integers of $n\in[1,p^k]$ so that $$\gcd(n,p^k)>1$$
What's more, if an integer and $p^k$ is not coprime, it can be divided by $p$ since all of prime factors of $p^k$ are $p$.

When $k=1$, we know $\varphi(p)=p-1$ since $p$ is a prime.

When $k=i$, suppose $\varphi(p^i)=p^{i}-p^{i-1}$.

When $k=i+1$, we know that there are $p^{i-1}$ integers in $[1,p^i]$ which are not coprime with $p^i$, so they are also not coprime with $p^{i+1}$. Then consider the integers $n\in[p^i+1,p^{i+1}]$ which are not coprime with $p^{i+1}$, we know that they all have a prime factor $p$, and $n/p\in[p^{i-1}+1,p^i]$, so there are $(p-1)p^{i-1}$ integers that satisfy this condition. In total, there are $p^{i-1}+(p-1)p^{i-1}=p^{i}$ integers which are not coprime with $p^{i+1}$, so $\varphi(p^{i+1})=p^{i+1}-p^i$.

According to the mathematical induction above, we can concluded that $$\varphi(p^k)=p^{k-1}(p-1)$$

\item
According to the Chinese Reminder Theorem, since $m$ and $n$ are coprime, there exists a ring isomorphism between $Z/mnZ$ and $Z/mZ\times Z/nZ$, and here $\varphi(mn)$ is the order of $Z/mnZ$, $\varphi(m)$ is the order of $Z/mZ$ and $\varphi(n)$ is the order of $Z/nZ$. Suppose $MN$ is the set of counted integers in $\varphi(mn)$, $M$ is that in $\varphi(M)$ and $N$ is that in $\varphi(N)$, there is a bijection between $MN$ and $M\times N$. So $\varphi(mn)=\varphi(m)\varphi(n)$.

\item
Suppose $$n=p_1^{k_1}p_2^{k_2}\cdots p_n^{k_n}$$ where $p_1,p_2,\cdots,p_n$ are primes and $k_1,k_2,\cdots,k_n\geqslant1$, it is obvious that $p_1^{k_1},p_2^{k_2},\cdots,p_n^{k_n}$ are pairwise coprime, so 
\begin{align*}
\varphi(n)&=\varphi(p_1^{k_1})\varphi(p_n^{k_n})\cdots\varphi(p_n^{k_n})\\
&=p_1^{k_1-1}(p_1-1)p_2^{k_2-1}(p_2-1)\cdots p_n^{k_n-1}(p_n-1)\\
&=p_1^{k_1}\left(1-\frac{1}{p_1}\right)p_2^{k_2}\left(1-\frac{1}{p_2}\right)\cdots p_n^{k_n}\left(1-\frac{1}{p_n}\right)\\
&=n\prod_{p|n}\left(1-\frac{1}{p}\right)
\end{align*}

\item
$$\varphi(1000)=1000\left(1-\frac{1}{2}\right)\left(1-\frac{1}{5}\right)=400$$

According to Euler's Theorem, since 7 is coprime with 1000, $$7^{400}\equiv1\mod1000$$
\begin{align*}
7^{803}&\equiv7^3\mod1000\\
&\equiv343\mod1000
\end{align*}
\end{enumerate}

\section{AES}
\begin{enumerate}
\item
128 bits of 1 is used as the key for round 1.
\item
$$K(5)=K(4)\oplus K(1)$$
\item
We know for a 4 bit number $X$, $$X\oplus 1111=\overline{X}$$
We also know $$K(0)=K(1)=K(2)=K(3)=1111$$
So it's easy to find
\begin{align*}
K(10)&=K(9)\oplus K(6)\\
&=[K(8)\oplus K(5)]\oplus[K(5)\oplus K(2)]\\
&=K(8)\oplus K(2)\\
&=\overline{K(8)}\\
K(11)&=K(10)\oplus K(7)\\
&=[K(9)\oplus K(6)]\oplus[K(6)\oplus K(3)]\\
&=K(9)\oplus K(3)\\
&=\overline{K(9)}
\end{align*}
\end{enumerate}

\section{Simple Questions}
\begin{enumerate}
\item
In ECB Mode, each block is encrypted separately with a function $E$ and a key $K$, so the corruption of one encrypted block won't influence other blocks, only one block will be decrypted incorrectly.

In CBC Mode, from the second block, each block is encrypted and xor with the previous encrypted block. If one encrypted block (not the last block) is corrupted, the next block will also be influenced when applied xor with the wrong block, so two blocks will be decrypted incorrectly.

\item



\item
Since $p=29$ is a prime, according to Theorem 2.17, we can test the prime factors of $p-1=28$, which are 2 and 7. 

First, when $q=2$, $$2^{(p-1)/q}=2^{28/2}=2^{14}\equiv28\mod 29$$
Second, when $q=7$, $$2^{(p-1)/q}=2^{28/7}=2^{4}\equiv16\mod 29$$
So $$2^{(p-1)/d}\not\equiv1\mod p$$
We can concluded that 2 is a generator of $U(Z/29Z)$.

\item
Since 1801 and 8191 are primes, it is a Legendre Symbol, and we can only directly calculate $1801^{4095}\mod8191$ to solve it.

By applying modular exponentiation, we get the following table.
\begin{center}
\begin{tabular}{ccc}
$i$ & $d_i$ & power mod 8191 \\\hline
11 & 1 & $1^2 \cdot 1801 \equiv 1801$ \\
10 & 1 & $1801^2 \cdot 1801 \equiv 2493$ \\
9 & 1 & $2493^2 \cdot 1801 \equiv 6873$ \\
8 & 1 & $6873^2 \cdot 1801 \equiv 7874$ \\
7 & 1 & $7874^2 \cdot 1801 \equiv 544$ \\
6 & 1 & $544^2 \cdot 1801 \equiv 557$ \\
5 & 1 & $557^2 \cdot 1801 \equiv 1193$ \\
4 & 1 & $1193^2 \cdot 1801 \equiv 4482$ \\
3 & 1 & $4482^2 \cdot 1801 \equiv 6085$ \\
2 & 1 & $6085^2 \cdot 1801 \equiv 5027$ \\
1 & 1 & $5027^2 \cdot 1801 \equiv 4046$ \\
0 & 1 & $4046^2 \cdot 1801 \equiv 8190$ \\

\end{tabular}
\end{center}
$$1801^{4095}\equiv8190\mod8191$$
$$\left(\frac{1801}{8191}\right)=-1$$

\item
First, if $\left(\frac{b^2-4ac}{p}\right)=0$, then $b^2-4ac=0$, so the equation only have one solution $x=-\frac{b}{2a}$, and it can always mod $p$, thus the number of solutions satisfies $1+\left(\frac{b^2-4ac}{p}\right)=1$.

Second, if $\left(\frac{b^2-4ac}{p}\right)\neq0$, then $b^2-4ac\neq0$, the equation have two solutions $x=-\frac{b\pm\sqrt{b^2-4ac}}{2a}$, which means
$$-\frac{b\pm\sqrt{b^2-4ac}}{2a}\equiv x\mod p$$
$$\sqrt{b^2-4ac}\equiv \pm(2ax+b)\mod p$$
Then the problem becomes whether $b^2-4ac$ is a square mod $p$.

If $\left(\frac{b^2-4ac}{p}\right)=1$, $b^2-4ac$ is a square mod $p$, and we can get 2 solutions mod $p$.

Otherwise, $\left(\frac{b^2-4ac}{p}\right)=-1$, $b^2-4ac$ is not a square mod $p$, and we can get no solution mod $p$.

In conclusion, the number of solutions mod $p$ to the equation $ax^2+bx+c$ is 
$$1+\left(\frac{b^2-4ac}{p}\right)$$
\item
According to Euler's theorem, $$n^{p-1}\equiv1\mod p$$ $$n^{q-1}\equiv1\mod q$$
Let $(p-1)=k(q-1)$,
$$(n^{q-1})^p=n^{p-1}\equiv1\mod q$$
Since $\gcd(n,pq)=1$, according to Chinese Reminder Theorem, we get
$$n^{p-1}\equiv1\mod pq$$

\item
If $\left(\dfrac{-3}{p}\right)=1$,
$$1\equiv(-3)^{(p-1)/2}\mod p$$
$$1\equiv3k\mod p,k\in Z$$



If $p\equiv1\mod3$, and $p$ is an odd prime, then $p\equiv1\mod6$.
$$x\equiv(-3)^{(p-1)/2}\mod p$$
$$x^2\equiv 1\mod p$$
And we know $(-3)^{(p-1)/2}=3k$, $k\in Z$, so
$$x\equiv 3k\mod p$$
\item

\end{enumerate}

\section{Prime vs. irreducible}

\section{Primitive root mod 65537}
\begin{enumerate}
\item
Since 65537 is a prime, we can calculate $3^{32768}\mod65537$ and we can find that $3^{32768}\equiv-1\mod65537$ (The calculation is shown in part 2), so $$\left(\frac{3}{65537}\right)=-1$$
\item
By applying modular exponentiation, we get the following table.
\begin{center}
\begin{tabular}{ccc}
$i$ & $d_i$ & power mod 65537 \\\hline
\begin{center}
\begin{tabular}{c|c|c}
$i$ & $d_i$ & power mod 11413 \\\hline 
12 & 1 & $1^2 \cdot 5859 \equiv 5859$ \\
11 & 1 & $5859^2 \cdot 5859 \equiv 1415$ \\
10 & 1 & $1415^2 \cdot 5859 \equiv 1617$ \\
9 & 0 & $1617^2 \equiv 1112$ \\
8 & 1 & $1112^2 \cdot 5859 \equiv 7374$ \\
7 & 0 & $7374^2 \equiv 4344$ \\
6 & 1 & $4344^2 \cdot 5859 \equiv 6768$ \\
5 & 1 & $6768^2 \cdot 5859 \equiv 4445$ \\
4 & 1 & $4445^2 \cdot 5859 \equiv 4041$ \\
3 & 1 & $4041^2 \cdot 5859 \equiv 11111$ \\
2 & 0 & $11111^2 \equiv 11313$ \\
1 & 1 & $11313^2 \cdot 5859 \equiv 7071$ \\
0 & 0 & $7071^2 \equiv 10101$ \\
\end{tabular}
\end{center}

\end{tabular}
\end{center}

So $$3^{32768}\equiv65536\mod65537$$
$$3^{32768}\not\equiv-1\mod65537$$

\item
According to Theorem 2.17, we can conclude that 3 is a primitive root mod 65537 because 2 is the only prime factor of 65536 and $3^{(65537-1)/2}\not\equiv1\mod65537$.
\end{enumerate}

\end{document}
