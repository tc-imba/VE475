\documentclass{article}
\usepackage{enumerate}
\usepackage{amsmath}
\usepackage{amssymb}
\usepackage{graphicx}
\usepackage{subfigure}
\usepackage{geometry}
\usepackage{caption}
\usepackage{indentfirst}
\usepackage{multirow}
\geometry{left=3.0cm,right=3.0cm,top=3.0cm,bottom=4.0cm}
\renewcommand{\thesection}{Ex. \arabic{section} ---}
\renewcommand{\mod}{{\rm\ mod\ }}
\title{VE475 Homework 4}
\author{Liu Yihao 515370910207}
\date{}

\begin{document}
\maketitle

\section{Euler's totient}
\begin{enumerate}
\item 
Suppose $$\varphi(p^k)=p^{k-1}(p-1)=p^k-p^{k-1}$$ which means, there are $p^{k-1}$ integers of $n\in[1,p^k]$ so that $$\gcd(n,p^k)>1$$
What's more, if an integer and $p^k$ is not coprime, it can be divided by $p$ since all of prime factors of $p^k$ are $p$.

When $k=1$, we know $\varphi(p)=p-1$ since $p$ is a prime.

When $k=i$, suppose $\varphi(p^i)=p^{i}-p^{i-1}$.

When $k=i+1$, we know that there are $p^{i-1}$ integers in $[1,p^i]$ which are not coprime with $p^i$, so they are also not coprime with $p^{i+1}$. Then consider the integers $n\in[p^i+1,p^{i+1}]$ which are not coprime with $p^{i+1}$, we know that they all have a prime factor $p$, and $n/p\in[p^{i-1}+1,p^i]$, so there are $(p-1)p^{i-1}$ integers that satisfy this condition. In total, there are $p^{i-1}+(p-1)p^{i-1}=p^{i}$ integers which are not coprime with $p^{i+1}$, so $\varphi(p^{i+1})=p^{i+1}-p^i$.

According to the mathematical induction above, we can concluded that $$\varphi(p^k)=p^{k-1}(p-1)$$

\item


\end{enumerate}

\end{document}
