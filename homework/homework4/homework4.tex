\documentclass{article}
\usepackage{enumerate}
\usepackage{amsmath}
\usepackage{amssymb}
\usepackage{graphicx}
\usepackage{subfigure}
\usepackage{geometry}
\usepackage{caption}
\usepackage{indentfirst}
\usepackage{multirow}
\geometry{left=3.0cm,right=3.0cm,top=3.0cm,bottom=4.0cm}
\renewcommand{\thesection}{Ex. \arabic{section} ---}
\renewcommand{\mod}{{\rm\ mod\ }}
\title{VE475 Homework 4}
\author{Liu Yihao 515370910207}
\date{}

\begin{document}
\maketitle

\section{Euler's totient}
\begin{enumerate}
\item 
Suppose $$\varphi(p^k)=p^{k-1}(p-1)=p^k-p^{k-1}$$ which means, there are $p^{k-1}$ integers of $n\in[1,p^k]$ so that $$\gcd(n,p^k)>1$$
What's more, if an integer and $p^k$ is not coprime, it can be divided by $p$ since all of prime factors of $p^k$ are $p$.

When $k=1$, we know $\varphi(p)=p-1$ since $p$ is a prime.

When $k=i$, suppose $\varphi(p^i)=p^{i}-p^{i-1}$.

When $k=i+1$, we know that there are $p^{i-1}$ integers in $[1,p^i]$ which are not coprime with $p^i$, so they are also not coprime with $p^{i+1}$. Then consider the integers $n\in[p^i+1,p^{i+1}]$ which are not coprime with $p^{i+1}$, we know that they all have a prime factor $p$, and $n/p\in[p^{i-1}+1,p^i]$, so there are $(p-1)p^{i-1}$ integers that satisfy this condition. In total, there are $p^{i-1}+(p-1)p^{i-1}=p^{i}$ integers which are not coprime with $p^{i+1}$, so $\varphi(p^{i+1})=p^{i+1}-p^i$.

According to the mathematical induction above, we can concluded that $$\varphi(p^k)=p^{k-1}(p-1)$$

\item
According to the Chinese Reminder Theorem, since $m$ and $n$ are coprime, there exists a ring isomorphism between $Z/mnZ$ and $Z/mZ\times Z/nZ$, and here $\varphi(mn)$ is the order of $Z/mnZ$, $\varphi(m)$ is the order of $Z/mZ$ and $\varphi(n)$ is the order of $Z/nZ$. Suppose $MN$ is the set of counted integers in $\varphi(mn)$, $M$ is that in $\varphi(M)$ and $N$ is that in $\varphi(N)$, there is a bijection between $MN$ and $M\times N$. So $\varphi(mn)=\varphi(m)\varphi(n)$.

\item
Suppose $$n=p_1^{k_1}p_2^{k_2}\cdots p_n^{k_n}$$ where $p_1,p_2,\cdots,p_n$ are primes and $k_1,k_2,\cdots,k_n\geqslant1$, it is obvious that $p_1^{k_1},p_2^{k_2},\cdots,p_n^{k_n}$ are pairwise coprime, so 
\begin{align*}
\varphi(n)&=\varphi(p_1^{k_1})\varphi(p_n^{k_n})\cdots\varphi(p_n^{k_n})\\
&=p_1^{k_1-1}(p_1-1)p_2^{k_2-1}(p_2-1)\cdots p_n^{k_n-1}(p_n-1)\\
&=p_1^{k_1}\left(1-\frac{1}{p_1}\right)p_2^{k_2}\left(1-\frac{1}{p_2}\right)\cdots p_n^{k_n}\left(1-\frac{1}{p_n}\right)\\
&=n\prod_{p|n}\left(1-\frac{1}{p}\right)
\end{align*}

\item
$$\varphi(1000)=1000\left(1-\frac{1}{2}\right)\left(1-\frac{1}{5}\right)=400$$

According to Euler's Theorem, since 7 is coprime with 1000, $$7^{400}\equiv1\mod1000$$
\begin{align*}
7^{803}&\equiv7^3\mod1000\\
&\equiv343\mod1000
\end{align*}
\end{enumerate}

\section{AES}
\begin{enumerate}
\item
128 bits of 1 is used as the key for round 1.
\item
$$K(5)=K(4)\oplus K(1)$$
\item
We know for a 4 bit number $X$, $$X\oplus 1111=\overline{X}$$
We also know $$K(0)=K(1)=K(2)=K(3)=1111$$
So it's easy to find
\begin{align*}
K(10)&=K(9)\oplus K(6)\\
&=[K(8)\oplus K(5)]\oplus[K(5)\oplus K(2)]\\
&=K(8)\oplus K(2)\\
&=\overline{K(8)}\\
K(11)&=K(10)\oplus K(7)\\
&=[K(9)\oplus K(6)]\oplus[K(6)\oplus K(3)]\\
&=K(9)\oplus K(3)\\
&=\overline{K(9)}
\end{align*}
\end{enumerate}

\section{Simple Questions}
\begin{enumerate}
\item

\end{enumerate}


\section{Primitive root mod 65537}
\begin{enumerate}
\item
Since 65537 is a prime, we can calculate $3^{32768}\mod65537$ and we can find that $3^{32768}\equiv-1\mod65537$ (The calculation is shown in part 2), so $$\left(\frac{3}{65537}\right)=-1$$
\item
By applying modular exponentiation, we get the following table.
\begin{center}
\begin{tabular}{ccc}
$i$ & $d_i$ & power mod 65537 \\\hline
15 & 1 & $1^2 \cdot 3 \equiv 3$ \\
14 & 0 & $3^2 \equiv 9$ \\
13 & 0 & $9^2 \equiv 81$ \\
12 & 0 & $81^2 \equiv 6561$ \\
11 & 0 & $6561^2 \equiv 54449$ \\
10 & 0 & $54449^2 \equiv 61869$ \\
9 & 0 & $61869^2 \equiv 19139$ \\
8 & 0 & $19139^2 \equiv 15028$ \\
7 & 0 & $15028^2 \equiv 282$ \\
6 & 0 & $282^2 \equiv 13987$ \\
5 & 0 & $13987^2 \equiv 8224$ \\
4 & 0 & $8224^2 \equiv 65529$ \\
3 & 0 & $65529^2 \equiv 64$ \\
2 & 0 & $64^2 \equiv 4096$ \\
1 & 0 & $4096^2 \equiv 65281$ \\
0 & 0 & $65281^2 \equiv 65536$ \\

\end{tabular}
\end{center}

So $$3^{32768}\equiv65536\mod65537$$
$$3^{32768}\not\equiv-1\mod65537$$

\item
According to Theorem 2.17, we can conclude that 3 is a primitive root mod 65537 because 2 is the only prime factor of 65536 and $3^{(65537-1)/2}\not\equiv1\mod65537$.
\end{enumerate}

\end{document}
