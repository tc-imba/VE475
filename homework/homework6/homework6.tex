\documentclass{article}
\usepackage{enumerate}
\usepackage{amsmath}
\usepackage{amssymb}
\usepackage{graphicx}
\usepackage{subfigure}
\usepackage{geometry}
\usepackage{caption}
\usepackage{indentfirst}
\usepackage{multirow}
\geometry{left=3.0cm,right=3.0cm,top=3.0cm,bottom=4.0cm}
\renewcommand{\thesection}{Ex. \arabic{section} ---}
\renewcommand{\mod}{{\rm\ mod\ }}
\title{VE475 Homework 6}
\author{Liu Yihao 515370910207}
\date{}

\begin{document}
\maketitle

\section{Application of the DLP}
\begin{enumerate}
\item
\begin{enumerate}[(a)]
\item
For Alice, she knows that $$\gamma\equiv\alpha^r\mod p$$
If Bob replies $$b\equiv r\mod p-1{\rm\ or\ }b\equiv x+r\mod p-1$$
She can get $$\alpha^{p-1}\equiv 1\mod p$$
$$\alpha^b\equiv\alpha^r\equiv\gamma\mod p{\rm\ or\ }\alpha^b\equiv\alpha^{x+r}\equiv\gamma\beta\mod p$$
So after calculating $\alpha^b\mod p$ and compare it with $\gamma$ or $\gamma\beta$, she can prove Bob's identity if he can calculate $x=\log_\alpha\beta$.
\item 
For Bob, if he doesn't know $x$, then he can't compute $b\equiv x+r\mod p-1$.If he want to know $x$, it becomes a DLP problem which is very difficult to solve, so he can prove his identity.
\end{enumerate}
\item
\begin{enumerate}[(a)]
\item 128 times.
\item 192 times.
\end{enumerate}
\item
It is Digital Signature Protocol.
\end{enumerate}

\section{Pohlig-Hellman}
First, let $g$ be a generator of the group, let $x=\log_gh$, let $n$ be the order of the group, obtain a prime factorization so that $$n=\prod_{i=1}^rp_i^{e_i}$$

Then, for each $i\in\{1,\dots,r\}$, compute $g_i=g^{n/p_i^{e_i}}$, which has order $p_i^{e_i}$, and compute $h_i=h^{n/p_i^{e_i}}$. Then we can use the Pohlig-Hellman algorithm for prime-power order to compute $x_i\in\{0,\dots,p_i^{e_1}-1\}$, which is described as follow:
\begin{enumerate}
\item Let $x=\log_gh$ ($x=x_i$, $g=g_i$, $h=h_i$ from previous part), where $g=p^e$, and first initialize $x_0=0$.
\item Set $\gamma=g^{p^{e-1}}$.
\item For each $k\in\{0,\dots,e-1\}$, compute $h_k=(g^{-x_k}h)^{p^{e-1-k}}$, By construction, the order of this element must divide $p$, hence $h_k\in\langle\gamma\rangle$. Then compute $d_k$ such that $\gamma^{d_k}=h_k$ and set $x_{k+1}=x_k+p^kd_k$.
\item Obtain $x=x_e$.
\end{enumerate}

After get all $x_i$, solve the simultaneous congruence $$x\equiv x_i \mod p_i^{e_i},i\in\{1,\dots,r\}$$

according to Chinese reminder theorem to get $x=\log_gh$.\\

As an example, we try to find $\log_3 3344$ in $G=U(Z/24389Z).$ Note that $24389=29^3$, so the order $n=28\cdot29^2=2^2\cdot7\cdot29^2$.

And 3 is a generator of $G$, so we can get 
\begin{align*}
g_1 &\equiv 3^{7\cdot29^2}\equiv 10133 \mod 24389\\
h_1 &\equiv 3344^{7\cdot29^2}\equiv 24388 \mod 24389\\
g_2 &\equiv 3^{2^2\cdot29^2}\equiv 7302 \mod 24389\\
h_2 &\equiv 3344^{2^2\cdot29^2}\equiv 4850 \mod 24389 \\
g_3 &\equiv 3^{2^2\cdot7}\equiv 11369 \mod 24389\\
h_3 &\equiv 3344^{2^2\cdot7}\equiv 23114 \mod 24389
\end{align*}

First, for $p=2$, $e=2$, $g=10133$ and $h=24388$, we should determine $x_a=\log_gh$. We can get
$$\gamma\equiv 10133^{2} \equiv 24388 \equiv -1 \mod 24389$$
$$
\begin{array}{ccc}
h_0 \equiv (10133^{0}\cdot -1)^{2} \equiv 1 \mod 24389, & d_0=0, & x_1\equiv 0 \mod 4 \\
h_1 \equiv (10133^{0}\cdot -1)^{1} \equiv -1 \mod 24389, & d_1=1, & x_2\equiv 2 \mod 4 \\
\end{array}
$$
$$x_a=2 \mod 4$$

Second, for $p=7$, $e=1$, $g=7302$ and $h=4850$, we should determine $x_b=\log_gh$. We can get
$$\gamma\equiv 7302^{1} \equiv 7302 \mod 24389$$
$$
\begin{array}{ccc}
h_0 \equiv (7302^{0}\cdot 4850)^{1} \equiv 4850 \mod 24389, & d_0=2, & x_1\equiv 2 \mod 7 \\
\end{array}
$$
$$x_b=2 \mod 7$$

Third, for $p=29$, $e=2$, $g=11369$ and $h=23114$, we should determine $x_c=\log_gh$. We can get
$$\gamma\equiv 11369^{29} \equiv 12616 \mod 24389$$
$$
\begin{array}{ccc}
h_0 \equiv (11369^{0}\cdot 23114)^{29} \equiv 11775 \mod 24389, & d_0=28, & x_1\equiv 28 \mod 841 \\
h_1 \equiv (11369^{-28}\cdot 23114)^{1} \equiv 3365 \mod 24389, & d_1=8, & x_2\equiv 260 \mod 841 \\
\end{array}
$$
$$x_c=260 \mod 841$$

According to Chinese remainder theorem, we can simply get 
\begin{align*}
x &\equiv 2 \mod 28 \\
x &\equiv 260 \mod 841 \\
841 \cdot 1 &\equiv 1 \mod 28 \\
28 \cdot 811 &\equiv 1 \mod 841 
\end{align*}
$$x\equiv 841 \cdot 1 \cdot 2 + 28 \cdot 811 \cdot 260 \equiv 18762 \mod 23548 $$


\section{Elgamal}
\begin{enumerate}
\item
If the polynomial $X^3+2X^2+1$ is reducible in $F_3[x]$, it can be factored as $$X^3+2X^2+1=(X+A)(X^2+BX+C)=X^3+A(B+1)X^2+(B+C)X+AC$$
There are two possible pairs of $(A,C)$, which are $(1,1)$ and $(2,2)$ so that $AC=1$.

First, if $A=C=1$, then $B=2$, but $A(B+1)=0\neq 2$, so it is wrong.

Second, if $A=C=2$, then $B=1$, but $A(B+1)=1\neq2$, so it is also wrong.

Then we can conclude that $X^3+2X^2+1$ is irreducible in $F_3[x]$.

According to Theorem 2.38, $X^3+2X^2+1$ is an irreducible polynomial of degree 3 in $F_3[x]$, let $F_{3^3}$ be the set of all the polynomial of degree less than 3 in $F_3[x]$, then $F_{3^3}$ is a finite field with $3^3=27$ elements.

\item

We can use 26 lower-case letters and define a map $\xi \leftrightarrow f(\xi)$, where $\xi$ is one of 26 letters. That is, $a \leftrightarrow 1$, $b \leftrightarrow 2$, \dots, $z \leftrightarrow 26$.

Let $P(x)=X^3+2X^2+1$,
\begin{center}
\begin{tabular}{|c|c|c|c|}
\hline & $q_i$ & $r_i$ & $s_i$ \\\hline
0 & & 7467 & 1 \\\hline
1 & & 11413 & 0 \\\hline
2 & $ 7467\div11413=0$ & $7467-0\times11413=7467$ & $1-0\times0=1$ \\\hline
3 & $ 11413\div7467=1$ & $11413-1\times7467=3946$ & $0-1\times1=-1$ \\\hline
4 & $ 7467\div3946=1$ & $7467-1\times3946=3521$ & $1-1\times-1=2$ \\\hline
5 & $ 3946\div3521=1$ & $3946-1\times3521=425$ & $-1-1\times2=-3$ \\\hline
6 & $ 3521\div425=8$ & $3521-8\times425=121$ & $2-8\times-3=26$ \\\hline
7 & $ 425\div121=3$ & $425-3\times121=62$ & $-3-3\times26=-81$ \\\hline
8 & $ 121\div62=1$ & $121-1\times62=59$ & $26-1\times-81=107$ \\\hline
9 & $ 62\div59=1$ & $62-1\times59=3$ & $-81-1\times107=-188$ \\\hline
10 & $ 59\div3=19$ & $59-19\times3=2$ & $107-19\times-188=3679$ \\\hline
11 & $ 3\div2=1$ & $3-1\times2=1$ & $-188-1\times3679=-3867$ \\\hline
\end{tabular}
\end{center}


So $X$ is a generator of $F_{3^3}$, and we can define the map as $$\xi \to g(\xi): g(\xi)=X^{f(\xi)} \mod P(X)$$ 

\item

According to Part 2, the order of the subgroup generated by $X$ is 26, 

\item
Use $X$ as the generator and 11 as the secret key,
$$X^{11}\equiv X-1 \equiv X+2 \mod P(X)$$
So $X+2$ is the public key.

\item
Choose $k=18$, we can get 
$$r \equiv X^{18} \equiv X+1 \mod P(X)$$
$$\beta^k \equiv (X+2)^{18} \equiv \mod P(X) $$

Then we can map the message ``goodmorning'' into $F_{3^3}$ as
\begin{center}
\begin{tabular}{c|c|c}
$i$ & $d_i$ & power mod 11413 \\\hline 
12 & 1 & $1^2 \cdot 5859 \equiv 5859$ \\
11 & 1 & $5859^2 \cdot 5859 \equiv 1415$ \\
10 & 1 & $1415^2 \cdot 5859 \equiv 1617$ \\
9 & 0 & $1617^2 \equiv 1112$ \\
8 & 1 & $1112^2 \cdot 5859 \equiv 7374$ \\
7 & 0 & $7374^2 \equiv 4344$ \\
6 & 1 & $4344^2 \cdot 5859 \equiv 6768$ \\
5 & 1 & $6768^2 \cdot 5859 \equiv 4445$ \\
4 & 1 & $4445^2 \cdot 5859 \equiv 4041$ \\
3 & 1 & $4041^2 \cdot 5859 \equiv 11111$ \\
2 & 0 & $11111^2 \equiv 11313$ \\
1 & 1 & $11313^2 \cdot 5859 \equiv 7071$ \\
0 & 0 & $7071^2 \equiv 10101$ \\
\end{tabular}
\end{center}


which can be encrypted by the equation
$$c \equiv \beta^k m \equiv (X+2)^{18} m \mod P(X) $$

The result $r$ is
$$X^2 + X,X,X,- X^2 + 1,- X^2 + X,X,X^2 - X - 1,1,- X^2 - X + 1,1,X^2 + X$$


Mapping them back to letters, we get the ciphertext ``saapyadzuzs''.

Then we can use $$m\equiv tr^{-x} \equiv t(X+1)^{-11} \mod P(X)$$

The result $m$ is
\begin{center}
\begin{tabular}{c|c|c}
$i$ & $d_i$ & power mod 11413 \\\hline 
12 & 1 & $1^2 \cdot 5859 \equiv 5859$ \\
11 & 1 & $5859^2 \cdot 5859 \equiv 1415$ \\
10 & 1 & $1415^2 \cdot 5859 \equiv 1617$ \\
9 & 0 & $1617^2 \equiv 1112$ \\
8 & 1 & $1112^2 \cdot 5859 \equiv 7374$ \\
7 & 0 & $7374^2 \equiv 4344$ \\
6 & 1 & $4344^2 \cdot 5859 \equiv 6768$ \\
5 & 1 & $6768^2 \cdot 5859 \equiv 4445$ \\
4 & 1 & $4445^2 \cdot 5859 \equiv 4041$ \\
3 & 1 & $4041^2 \cdot 5859 \equiv 11111$ \\
2 & 0 & $11111^2 \equiv 11313$ \\
1 & 1 & $11313^2 \cdot 5859 \equiv 7071$ \\
0 & 0 & $7071^2 \equiv 10101$ \\
\end{tabular}
\end{center}


So the plaintext is successfully decrypted.
\end{enumerate}

\section{Simple Questions}
\begin{enumerate}
\item
\begin{enumerate}[(i)]
\item Yes. We know $h(x)\equiv x^2 \mod pq$, and we can find $x$ by computing $\sqrt{h(x)}\mod p$ and $\sqrt{h(x)}\mod q$ and then use Chinese remainder theorem. However, $p,q$ are large primes, the factorization of $n$ is very difficult, so we can't efficiently find $x$.
\item No. Given $x$, we can find $x'=-x$ so that $h(x)=h(x')$.
\item No. For any $x$ and $x'$ so that $x'=-x$, we can find $h(x)=h(x')$.
\end{enumerate}
\item
\begin{enumerate}[(i)]
\item Efficiently computed for any input can be verified. Any length of message $m$ can be computed into an 160 bits length result efficiently through xor.
\item Pre-image resistant is not verified. Given $y$, let $m=y$, we can get $h(m)=y$. 
\item Second pre-image resistant is not verified. Given $m$, we can add 160 bits 0 after $m$ to form $m'$, so that $h(m)=h(m')$.
\item Collision resistant is not verified. For any $m$ and $m'$ so that 160 bits 0 after $m$ are added to to form $m'$, we can find $h(m)=h(m')$.
\end{enumerate}
\end{enumerate}

\section{Merkle-Damgård construction}
\begin{enumerate}
\item
\begin{enumerate}[a)]
\item
Suppose the map $s$ is not injective, that is, $\exists x\neq x'$ so that $y=y'$. Than we can apply the following strategy to examine. Let $y_0=y$, if $y_{0,|y_0|-1}||y_{0,|y_0|}=01$, we can find $x_{|x|}=x'_{|x'|}=1$, and let $y_0=y_{0,1}||\cdots||y_{0,|y_0|-2}$. Otherwise, if $y_{0,|y_0|}=0$, we can find $x_{|x|}=x'_{|x'|}=0$, and let $y_0=y_{0,1}||\cdots||y_{0,|y_0|-1}$. Repeating the strategy until $|y_0|=11$, we can find all bits of $x$ and $x'$ are the same, so $x=x'$, which makes a contradiction. So map $s$ is injective.
\item
If $z$ is empty, according to a), we know there is no strings $x\neq x'$ and $z$ such that $s(x) = z||s(x')$.\\
If $z$ is not empty, let $a=z||s(x')$, we can find a substring 11 in $a_1||a_2||\cdots||a_{|a|}$. However, we can only find 11 in $s(x)_0||s(x)_1$, which makes a contradiction.\\
So we can conclude that there is no strings $x\neq x'$ and $z$ such that $s(x) = z||s(x')$.
\end{enumerate}
\item
From the two previous conditions, we know collisions can't be found through changing bits of input or adding paddings, which means the map $s$ is collision resistant.
\item

\end{enumerate}

\section{Programming}
In the ex6 folder, with a README file inside it.

\end{document}
