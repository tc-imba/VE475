\documentclass{article}
\usepackage{enumerate}
\usepackage{amsmath}
\usepackage{amssymb}
\usepackage{graphicx}
\usepackage{subfigure}
\usepackage{geometry}
\usepackage{caption}
\usepackage{indentfirst}
\usepackage{multirow}
\geometry{left=3.0cm,right=3.0cm,top=3.0cm,bottom=4.0cm}
\renewcommand{\thesection}{Ex. \arabic{section} ---}
\renewcommand{\mod}{{\rm\ mod\ }}
\title{VE475 Homework 6}
\author{Liu Yihao 515370910207}
\date{}

\begin{document}
\maketitle

\section{Application of the DLP}
\begin{enumerate}
\item
\begin{enumerate}[(a)]
\item
For Alice, she knows that $$\gamma\equiv\alpha^r\mod p$$
If Bob replies $$b\equiv r\mod p-1{\rm\ or\ }b\equiv x+r\mod p-1$$
She can get $$\alpha^{p-1}\equiv 1\mod p$$
$$\alpha^r\equiv\alpha^b\equiv\gamma\mod p{\rm\ or\ }\alpha^r\equiv\alpha^{b-x}\equiv\gamma\mod p$$
So after calculating $\alpha^b\mod p$ or $\alpha^{b-x}\mod p$ and compare it with $\gamma$, she can prove Bob's identity if he replies the correct $b$.
\item 
For Bob, he doesn't know $r$, but he can compute $b=\log_\alpha\gamma$ or $b=\log_\alpha\gamma+x$ so that $b\equiv r\mod p-1$. If he can't do so, it becomes a DLP problem which is very difficult to solve, so he can prove his identity.
\end{enumerate}
\item
\begin{enumerate}[(a)]
\item
\item
\end{enumerate}
\item
It is Digital Signature Protocol.
\end{enumerate}

\end{document}
