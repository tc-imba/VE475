\documentclass{article}
\usepackage{enumerate}
\usepackage{amsmath}
\usepackage{amssymb}
\usepackage{graphicx}
\usepackage{subfigure}
\usepackage{geometry}
\usepackage{caption}
\usepackage{indentfirst}
\usepackage{multirow}
\geometry{left=3.0cm,right=3.0cm,top=3.0cm,bottom=4.0cm}
\renewcommand{\thesection}{Ex. \arabic{section} ---}
\renewcommand{\mod}{{\rm\ mod\ }}
\title{VE475 Homework 3}
\author{Liu Yihao 515370910207}
\date{}

\begin{document}
\maketitle

\section{Finite fields}
\begin{enumerate}
\item
The possible factors of $X^2+1$ in $F_3[X]$ are $X,X+1,X+2$ 
$$X(X+1)=X^2+X\neq X^2+1$$
$$X(X+2)=X^2+2X\neq X^2+1$$
$$(X+1)(X+2)=X^2+3X+2=X^2+2\neq X^2+1$$
$$X\cdot X=X^2\neq X^2+1$$
$$(X+1)(X+1)=X^2+2X+1\neq X^2+1$$
$$(X+2)(X+2)=X^2+4X+4=X^2+X+1\neq X^2+1$$
So $X^2+1$ is irreducible in $F_3[X]$
\item
According to the Proof on c2, Page 39, if $P(X)$ is irreducible and $A(X)$ is a polynomial in a finite field, there exists a polynomial $B(X)$ such that $$A(X)B(X)\equiv1\mod P(X)$$
Here let $P(X)=X^2+1$, $A(X)=1+2X$, then $B(X)$ is the multiplication inverse of $1+2X\mod X^2+1$.
\item
Applying the Extended Euclid Algorithm,
\begin{center}
\begin{tabular}{|c|c|c|c|c|}
\hline
& $q_i$ & $r_i$ & $s_i$ & $t_i$\\\hline
0 & & $2X+1$ & 1 & 0\\\hline
1 & & $X^2+1$ & 0 & 1\\\hline
2 & $(2X+1)\div(X^2+1)=0$ & $2X+1$ & $1$ & $0$\\\hline
3 & $(X^2+1)\div(2X+1)=2X$ & $X+1$ & $X$ & $1$\\\hline
4 & $(2X+1)\div(X+1)=2$ & $2$ & $X+1$ & $1$\\\hline
5 & $(X+1)\div2=2X$ & $1$ & $X^2+2X$ & $X+1$\\\hline
\end{tabular}
\end{center}
$$(1+2X)(X^2+2X)\equiv1\mod X^2+1$$
\end{enumerate}

\section{AES}
\begin{enumerate}
\item
\begin{enumerate}[(a)]
\item
\emph{InvShiftRows} cyclically shift to the right row $i$ by offset $i$, $0\leqslant i\leqslant3$.
\item
The inverse of \emph{AddRoundKey} is actually the same as itself, since if we xor a value by another value twice, it will keep not changed. We only need to reverse the order of round keys.
\item
The transformation matrix of \emph{MixColumns} is 
$$A=\begin{pmatrix}
00000010 & 00000011 & 00000001 & 00000001 \\
00000001 & 00000010 & 00000011 & 00000001 \\
00000001 & 00000001 & 00000010 & 00000011 \\
00000011 & 00000001 & 00000001 & 00000010 \\
\end{pmatrix}=\begin{pmatrix}
02 & 03 & 01 & 01 \\
01 & 02 & 03 & 01 \\
01 & 01 & 02 & 03 \\
03 & 01 & 01 & 02 \\
\end{pmatrix}$$
If the transformation matrix of \emph{InvMixColumns} is 
$$B=\begin{pmatrix}
00001110 & 00001011 & 00001101 & 00001001 \\
00001001 & 00001110 & 00001011 & 00001101 \\
00001101 & 00001001 & 00001110 & 00001011 \\
00001011 & 00001101 & 00001001 & 00001110 \\
\end{pmatrix}=\begin{pmatrix}
0E & 0B & 0D & 09 \\
09 & 0E & 0B & 0D \\
0D & 09 & 0E & 0B \\
0B & 0D & 09 & 0E \\
\end{pmatrix}$$

We can calculate $BA$ according to the definition of $GF(2^8)$.\\
For example (in hex form), in the first column, 
\begin{align*}
(0E\cdot02)\oplus(0B\cdot01)\oplus(0D\cdot01)\oplus(09\cdot03)&=01\\
(09\cdot02)\oplus(0E\cdot01)\oplus(0B\cdot01)\oplus(0D\cdot03)&=00\\
(0D\cdot02)\oplus(09\cdot01)\oplus(0E\cdot01)\oplus(0B\cdot03)&=00\\
(0B\cdot02)\oplus(0D\cdot01)\oplus(09\cdot01)\oplus(0E\cdot03)&=00
\end{align*}
The calculation of other three column is similar, thus we can get
$$BA=\begin{pmatrix}
01 & 00 & 00 & 00 \\
00 & 01 & 00 & 00 \\
00 & 00 & 01 & 00 \\
00 & 00 & 00 & 01 \\
\end{pmatrix}=\begin{pmatrix}
00000001 & 00000000 & 00000000 & 00000000 \\
00000000 & 00000001 & 00000000 & 00000000 \\
00000000 & 00000000 & 00000001 & 00000000 \\
00000000 & 00000000 & 00000000 & 00000001 \\
\end{pmatrix}=I$$
If the origin matrix is $S$, the mix-columned matrix is $AS$, then $$B(AS)=BA(S)=IS=S$$
\end{enumerate}
\item
First, we generate the round keys according to the key, then we apply \emph{AddRoundKey} with round key (40--43).\\[1mm]
Second, we apply nine turns of following four steps ($i$ is the turn number): \emph{InvShiftRows}, \emph{InvSubBytes}, \emph{AddRoundKey} with round key ($40-4*i$--$43-4*i$) and \emph{InvMixColumns}.\\[1mm]
At last, we apply \emph{InvShiftRows}, \emph{InvSubBytes} and \emph{AddRoundKey} with round key (0--3).
\item
Since \emph{InvShiftRows} doesn't change the value of any cell, and \emph{InvSubBytes} only substitutes the value of each cell according to a table, the order of applying them doesn't influence the result. So they can be applied on reverse order.
\item
\begin{enumerate}[(a)]
\item
Since \emph{InvMixColumns} and \emph{AddRoundKey} affect the value of each column based on completely different theorems, the reverse order may cause a different result.
\item
$$[(m_{i,j})(a_{i,j})]\oplus(k_{i,j})$$
\item
$$(a_{i,j})=(m_{i,j})^{-1}[(e_{i,j})\oplus(k_{i,j})]=[(m_{i,j})^{-1}(e_{i,j})]\oplus[(m_{i,j})^{-1}(k_{i,j})]$$
So the inverse operation is given by
$$(e_{i,j})\longrightarrow(m_{i,j})^{-1}(e_{i,j})\oplus(m_{i,j})^{-1}(k_{i,j})$$
\item
\emph{InvAddRoundKey} first apply \emph{InvMixColumns} to the key, then apply \emph{AddRoundKey} to the data with the inv-mix-columned key.
\end{enumerate}
\item
First, we generate the round keys according to the key, then we apply \emph{AddRoundKey} with round key (40--43).\\[1mm]
Second, we apply nine turns of following four steps ($i$ is the turn number): \emph{InvSubBytes}, \emph{InvShiftRows}, \emph{InvMixColumns} and \emph{InvAddRoundKey} with round key ($40-4*i$--$43-4*i$).\\[1mm]
At last, we apply \emph{InvSubBytes}, \emph{InvShiftRows} and \emph{AddRoundKey} with round key (0--3).
\item The advantage of this strategy is that we only need to implement the four inverse transformations and apply them in the same order as the encryption process. Thus the encryption and decryption process can be unified and written only once.
\end{enumerate}

\section{DES}
\begin{enumerate}
\item
In DES, the length of plaintext is 64 bits, and the length of key is 56 bits.

First, the plaintext is transformed according to the following table:
\begin{center}
\begin{tabular}{c|cccccccc}
& 1 & 2 & 3 & 4 & 5 & 6 & 7 & 8 \\
\hline
1 & 58 & 50 & 42 & 34 & 26 & 18 & 10 & 2 \\
2 & 60 & 52 & 44 & 36 & 28 & 20 & 12 & 4 \\
3 & 62 & 54 & 46 & 38 & 30 & 22 & 14 & 6 \\
4 & 64 & 56 & 48 & 40 & 32 & 24 & 16 & 8 \\
5 & 57 & 49 & 41 & 33 & 25 & 17 &  9 & 1 \\
6 & 59 & 51 & 43 & 35 & 27 & 19 & 11 & 3 \\
7 & 61 & 53 & 45 & 37 & 29 & 21 & 13 & 5 \\
8 & 63 & 55 & 47 & 39 & 31 & 23 & 15 & 7 \\
\end{tabular}
\end{center}

Then the transformed plaintext is divided into two parts of 32 bits, and we apply Feistel Network for 16 turns, each turn can be expressed as
$$L_i=R_{i-1}$$
$$R_i=L_{i-1}\oplus F(R_{i-1},K_i)$$

The procedure of function $F(R_{i-1},K_i)$ is
\begin{enumerate}[(a)]
\item
Expand $R_{i-1}$ (32 bits) into 48 bits a according to the following table:
\begin{center}
\begin{tabular}{c|cccccc}
& 1 & 2 & 3 & 4 & 5 & 6 \\
\hline
1 & 32 & 1 & 2 & 3 & 4 & 5 \\
2 & 4 & 5 & 6 & 7 & 8 & 9 \\
3 & 8 & 9 & 10 & 11 & 12 & 13 \\
4 & 12 & 13 & 14 & 15 & 16 & 17 \\
5 & 16 & 17 & 18 & 19 & 20 & 21 \\
6 & 20 & 21 & 22 & 23 & 24 & 25 \\
7 & 24 & 25 & 26 & 27 & 28 & 29 \\
8 & 28 & 29 & 30 & 31 & 32 & 1 \\
\end{tabular}
\end{center}
\item
Xor the 48 bits data and the key $K_i$
\item
Reduce the 48 bits data into 32 bits with eight S-boxes. Each S-box accept 6 bits and output 4 bits.
\begin{center}
\begin{tabular}{c|cccccccccccccccc}
\multirow{4}{*}{$S_1$} & 14 & 4 & 13 & 1 & 2 & 15 & 11 & 8 & 3 & 10 & 6 & 12 & 5 & 9 & 0 & 7 \\
& 0 & 15 & 7 & 4 & 14 & 2 & 13 & 1 & 10 & 6 & 12 & 11 & 9 & 5 & 3 & 8 \\  
& 4 & 1 & 14 & 8 & 13 & 6 & 2 & 11 & 15 & 12 & 9 & 7 & 3 & 10 & 5 & 0 \\  
& 15 & 12 & 8 & 2 & 4 & 9 & 1 & 7 & 5 & 11 & 3 & 14 & 10 & 0 & 6 & 13 \\\hline  
\multirow{4}{*}{$S_2$} & 15 & 1 & 8 & 14 & 6 & 11 & 3 & 4 & 9 & 7 & 2 & 13 & 12 & 0 & 5 & 10 \\  
& 3 & 13 & 4 & 7 & 15 & 2 & 8 & 14 & 12 & 0 & 1 & 10 & 6 & 9 & 11 & 5 \\  
& 0 & 14 & 7 & 11 & 10 & 4 & 13 & 1 & 5 & 8 & 12 & 6 & 9 & 3 & 2 & 15 \\  
& 13 & 8 & 10 & 1 & 3 & 15 & 4 & 2 & 11 & 6 & 7 & 12 & 0 & 5 & 14 & 9 \\\hline
\multirow{4}{*}{$S_3$} & 10 & 0 & 9 & 14 & 6 & 3 & 15 & 5 & 1 & 13 & 12 & 7 & 11 & 4 & 2 & 8 \\  
& 13 & 7 & 0 & 9 & 3 & 4 & 6 & 10 & 2 & 8 & 5 & 14 & 12 & 11 & 15 & 1 \\  
& 13 & 6 & 4 & 9 & 8 & 15 & 3 & 0 & 11 & 1 & 2 & 12 & 5 & 10 & 14 & 7 \\  
& 1 & 10 & 13 & 0 & 6 & 9 & 8 & 7 & 4 & 15 & 14 & 3 & 11 & 5 & 2 & 12 \\\hline
\multirow{4}{*}{$S_4$} & 7 & 13 & 14 & 3 & 0 & 6 & 9 & 10 & 1 & 2 & 8 & 5 & 11 & 12 & 4 & 15 \\  
& 13 & 8 & 11 & 5 & 6 & 15 & 0 & 3 & 4 & 7 & 2 & 12 & 1 & 10 & 14 & 9 \\  
& 10 & 6 & 9 & 0 & 12 & 11 & 7 & 13 & 15 & 1 & 3 & 14 & 5 & 2 & 8 & 4 \\  
& 3 & 15 & 0 & 6 & 10 & 1 & 13 & 8 & 9 & 4 & 5 & 11 & 12 & 7 & 2 & 14 \\\hline                                                                     
\multirow{4}{*}{$S_5$} & 2 & 12 & 4 & 1 & 7 & 10 & 11 & 6 & 8 & 5 & 3 & 15 & 13 & 0 & 14 & 9 \\  
& 14 & 11 & 2 & 12 & 4 & 7 & 13 & 1 & 5 & 0 & 15 & 10 & 3 & 9 & 8 & 6 \\  
& 4 & 2 & 1 & 11 & 10 & 13 & 7 & 8 & 15 & 9 & 12 & 5 & 6 & 3 & 0 & 14 \\  
& 11 & 8 & 12 & 7 & 1 & 14 & 2 & 13 & 6 & 15 & 0 & 9 & 10 & 4 & 5 & 3 \\\hline                                                                     
\multirow{4}{*}{$S_6$} & 12 & 1 & 10 & 15 & 9 & 2 & 6 & 8 & 0 & 13 & 3 & 4 & 14 & 7 & 5 & 11 \\  
& 10 & 15 & 4 & 2 & 7 & 12 & 9 & 5 & 6 & 1 & 13 & 14 & 0 & 11 & 3 & 8 \\  
& 9 & 14 & 15 & 5 & 2 & 8 & 12 & 3 & 7 & 0 & 4 & 10 & 1 & 13 & 11 & 6 \\  
& 4 & 3 & 2 & 12 & 9 & 5 & 15 & 10 & 11 & 14 & 1 & 7 & 6 & 0 & 8 & 13 \\\hline                                                                     
\multirow{4}{*}{$S_7$} & 4 & 11 & 2 & 14 & 15 & 0 & 8 & 13 & 3 & 12 & 9 & 7 & 5 & 10 & 6 & 1 \\  
& 13 & 0 & 11 & 7 & 4 & 9 & 1 & 10 & 14 & 3 & 5 & 12 & 2 & 15 & 8 & 6 \\  
& 1 & 4 & 11 & 13 & 12 & 3 & 7 & 14 & 10 & 15 & 6 & 8 & 0 & 5 & 9 & 2 \\  
& 6 & 11 & 13 & 8 & 1 & 4 & 10 & 7 & 9 & 5 & 0 & 15 & 14 & 2 & 3 & 12 \\\hline                                                                     
\multirow{4}{*}{$S_8$} & 13 & 2 & 8 & 4 & 6 & 15 & 11 & 1 & 10 & 9 & 3 & 14 & 5 & 0 & 12 & 7 \\  
& 1 & 15 & 13 & 8 & 10 & 3 & 7 & 4 & 12 & 5 & 6 & 11 & 0 & 14 & 9 & 2 \\  
& 7 & 11 & 4 & 1 & 9 & 12 & 14 & 2 & 0 & 6 & 10 & 13 & 15 & 3 & 5 & 8 \\  
& 2 & 1 & 14 & 7 & 4 & 10 & 8 & 13 & 15 & 12 & 9 & 0 & 3 & 5 & 6 & 11 \\  
\end{tabular}
\end{center}
\item
At last, we apply another transformation to the 32 bits data according to the following table:
\begin{center}
\begin{tabular}{c|cccccccc}
& 1 & 2 & 3 & 4 & 5 & 6 & 7 & 8 \\
\hline
1 & 16 & 7 & 20 & 21 & 29 & 12 & 28 & 17 \\
2 & 1 & 15 & 23 & 26 & 5 & 18 & 31 & 10 \\
3 & 2 & 8 & 24 & 14 & 32 & 27 & 3 & 9 \\
4 & 19 & 13 & 30 & 6 & 22 & 11 & 4 & 25 \\
\end{tabular}
\end{center}
\end{enumerate}

The round keys $K_i$ are generated in this method:
\begin{enumerate}[(a)]
\item
First, transform the 56 bits key $K$ according to the following table and divide the result into two 28 bits data $C_0$ and $D_0$.
\begin{center}
\begin{tabular}{c|ccccccc}
& 1 & 2 & 3 & 4 & 5 & 6 & 7 \\\hline
1 & 57 & 49 & 41 & 33 & 25 & 17 & 9 \\
2 & 1 & 58 & 50 & 42 & 34 & 26 & 18 \\
3 & 10 & 2 & 59 & 51 & 43 & 35 & 27 \\
4 & 19 & 11 & 3 & 60 & 52 & 44 & 36 \\
5 & 63 & 55 & 47 & 39 & 31 & 23 & 15 \\
6 & 7 & 62 & 54 & 46 & 38 & 30 & 22 \\
7 & 14 & 6 & 61 & 53 & 45 & 37 & 29 \\
8 & 21 & 13 & 5 & 28 & 20 & 12 & 4 \\
\end{tabular}
\end{center}
\item
Left shift $C_{i-1}$ and $D_{i-1}$ according to the following table to get $C_i$ and $D_i$, where $1\leqslant i\leqslant16$.
\begin{center}
\begin{tabular}{c|cccccccccccccccc}
$i$ & 1 & 2 & 3 & 4 & 5 & 6 & 7 & 8 & 9 & 10 & 11 & 12 & 13 & 14 & 15 & 16 \\\hline
shift & 1 & 1 & 2 & 2 & 2 & 2 & 2 & 2 & 1 & 2 & 2 & 2 & 2 & 2 & 2 & 1
\end{tabular}
\end{center}
\item
Concat 28 bits $C_i$ and $D_i$ and transform it according to the following table, which forms 48 bits $K_i$.
\begin{center}
\begin{tabular}{c|cccccccc}
& 1 & 2 & 3 & 4 & 5 & 6 & 7 & 8 \\\hline
1 & 14 & 17 & 11 & 24 & 1 & 5 & 3 & 28 \\
2 & 15 & 6 & 21 & 10 & 23 & 19 & 12 & 4 \\
3 & 26 & 8 & 16 & 7 & 27 & 20 & 13 & 2 \\
4 & 41 & 52 & 31 & 37 & 47 & 55 & 30 & 40 \\
5 & 51 & 45 & 33 & 48 & 44 & 49 & 39 & 56 \\
6 & 34 & 53 & 46 & 42 & 50 & 36 & 29 & 32 \\
\end{tabular}
\end{center}
\end{enumerate}

At last, we apply a reverse transformation, which is shown in the following table.
\begin{center}
\begin{tabular}{c|cccccccc}
& 1 & 2 & 3 & 4 & 5 & 6 & 7 & 8 \\
\hline
1 & 40 & 8 & 48 & 16 & 56 & 24 & 64 & 32 \\
2 & 39 & 7 & 47 & 15 & 55 & 23 & 63 & 31 \\
3 & 38 & 6 & 46 & 14 & 54 & 22 & 62 & 30 \\
4 & 37 & 5 & 45 & 13 & 53 & 21 & 61 & 29 \\
5 & 36 & 4 & 44 & 12 & 52 & 20 & 60 & 28 \\
6 & 35 & 3 & 43 & 11 & 51 & 19 & 59 & 27 \\
7 & 34 & 2 & 42 & 10 & 50 & 18 & 58 & 26 \\
8 & 33 & 1 & 41 &  9 & 49 & 17 & 57 & 25 \\
\end{tabular}
\end{center}

The decryption method is actually the same as the encryption method, we need only reverse the order of $K_i$.

\item
Linear cryptanalysis:

There are two parts to linear cryptanalysis. The first is to construct linear equations relating plaintext, ciphertext and key bits whose probabilities of holding are close to 0 or 1. The second is to use these linear equations in conjunction with known plaintext-ciphertext pairs to derive key bits.

Differential cryptanalysis:

The attack discover whether a given input/output difference pattern only occurs for certain values of inputs. Observing the desired output difference (between two chosen or known plaintext inputs) suggests possible key values.

\item
The safety of double DES is similar to single DES, since it suffers meet-in-the-middle attack. suppose the encryption map relation is $X\rightarrow E(K,X)$, the decryption map relation is $X\rightarrow D(K,X)$, the plaintext is $P$, the ciphertext is $C$, then
$$C=E(K_2,E(K_1,P))$$
Suppose the attacker knows a pair of $P$ and $C$, $\exists X$ such that $$X=E(K_1,P)=D(K_2,C)$$
First, he can try to encrypt $P$ with all of $2^{56}$ values of $K_1$ into $2^{56}$ values of $X_1$. Then he can decode $C$ with all of $2^{56}$ values of $K_2$ into $X_2$ until he find that $X_2$ is identical to one of $X_1$. In this situation, there are $2^{112}$ possible combinations of $K_1,K_2$, so there are $2^{112}/2^{64}=2^{48}$ wrong results.

However, if the attacker knows two pairs of $P$ and $C$, the possibility of wrong result decreases to $2^{48}/2^{64}=2^{-16}$, which is very small. So we can concluded that if a key pair $K_1,K_2$ satisfy both two pairs of $P$ and $C$, they are the actual keys.

With this method, a double DES can be broken in $2^{57}$ operations, which is not considered safe anymore in modern cryptography.

If we use triple DES, we can avoid this kind of attack. Suppose we also use two keys $K_1$ and $K_2$ to apply the encryption and decryption procedure:
$$C=E(K_1,D(K_2,E(K_1,P)))$$
$$P=D(K_1,E(K_2,D(K_1,P)))$$

People often use this kind of encryption and decryption method in triple DES because if $K_1=K_2$, the algorithm behaves same as a single DES, which can provide compatibility.

If meet-in-the-middle is applied to triple DES, the time complexity becomes $2^{112}$ to encrypt, which can be considered safe anymore in modern cryptography.

\item
According to the \emph{man passwd}, The legacy UNIX System encryption method is based on the NBS DES algorithm. User's password will be encrypted and saved to \emph{/etc/passwd}. However, it is a problem that single DES is no longer safe now. So for security, modern Unix based system use more recent methods, such as SHA, to validate the password.

\end{enumerate}

\section{Programming}
In the ex4 folder, with a README file inside it.

\end{document}
