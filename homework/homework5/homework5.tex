\documentclass{article}
\usepackage{enumerate}
\usepackage{amsmath}
\usepackage{amssymb}
\usepackage{graphicx}
\usepackage{subfigure}
\usepackage{geometry}
\usepackage{caption}
\usepackage{indentfirst}
\usepackage{multirow}
\geometry{left=3.0cm,right=3.0cm,top=3.0cm,bottom=4.0cm}
\renewcommand{\thesection}{Ex. \arabic{section} ---}
\renewcommand{\mod}{{\rm\ mod\ }}
\title{VE475 Homework 5}
\author{Liu Yihao 515370910207}
\date{}

\begin{document}
\maketitle

\section{RSA setup}
\begin{enumerate}
\item
In the RSA encryption and decryption, we use $$ed\equiv1\mod\varphi(n)$$
$$m^{ed}\equiv m\mod\varphi(n)$$
This is based on the Euler's theorem, which has a condition that $m$ and $n$ be two coprime integers. So it is likely for $n$ to be coprime with $m$.
\item
Suppose $k=a\varphi(n)$, $a\in N^*$, and $m<n$.
\begin{enumerate}[(a)]
\item
\begin{align*}
m^k&\equiv(m^{\varphi(n)})^a\mod n\\
&\equiv1^a\mod n\\
&\equiv1\mod n\\
\end{align*}
So $$m^k\equiv1\mod p\quad{\rm and}\quad m^k\equiv1\mod q$$
\item
First, if $\gcd(m,n)=1$, according to (a), it's obvious that
$$m^{k+1}\equiv m\mod p\quad{\rm and}\quad m^{k+1}\equiv m\mod q$$
Second, if $\gcd(m,n)=p$, so $\gcd(m/p,q)=1$
\begin{align*}
m^{k+1}&\equiv p\left[\left(\frac{m}{p}\right)^{k+1}\mod q\right]\mod n\\
&\equiv p\left[\left(\frac{m}{p}\right)^{a(p-1)\varphi(q)+1}\mod q\right]\mod n\\
&\equiv p\cdot\frac{m}{p}\mod n\\
&\equiv m\mod n
\end{align*}
So $$m^{k+1}\equiv m\mod p\quad{\rm and}\quad m^{k+1}\equiv m\mod q$$
Third, if $\gcd(m,n)=q$, it is similar to the second case.

We can conclude that for any arbitrary $m$, $m^{k+1}\equiv m\mod p{\rm\ and}\mod q$.
\end{enumerate}
\item
\begin{enumerate}[(a)]
\item
We know that $ed\equiv1\mod\varphi(n)$, which means that $ed=k+1$ where $k$ is a multiple of $\varphi(n)$. According to part 2(b), we know that for any arbitrary $m$, $m^{k+1}\equiv m\mod p{\rm\ and}\mod q$, or we can say $m^{k+1}\equiv m\mod n$, so $m^{ed}\equiv m\mod n$,
\item
From the previous calculation, we can find that for all $m<n$, no matter $m$ and $n$ are coprime or not, we can both find that $m^{ed}\equiv m\mod n$, so that the RSA encryption and decryption can be performed. So we can conclude that it is not necessary that $\gcd(m,n)=1$.
\end{enumerate}
\end{enumerate}

\section{RSA decryption}
$$n=11413=101\times113$$

So we can find that $p=101$ and $q=113$, so $\varphi(n)=11200$, and we should calculate $d$ so that $ed\equiv1\mod \varphi(n)$.\\[-0.5em]

By applying the extended euclidean algorithm,
\begin{center}
\begin{tabular}{c|c|c}
$i$ & $d_i$ & power mod 65537 \\\hline 
11 & 1 & $1^2 \cdot 3 \equiv 3$ \\
10 & 0 & $3^2 \equiv 9$ \\
9 & 0 & $9^2 \equiv 81$ \\
8 & 0 & $81^2 \equiv 6561$ \\
7 & 0 & $6561^2 \equiv 54449$ \\
6 & 0 & $54449^2 \equiv 61869$ \\
5 & 0 & $61869^2 \equiv 19139$ \\
4 & 0 & $19139^2 \equiv 15028$ \\
3 & 0 & $15028^2 \equiv 282$ \\
2 & 0 & $282^2 \equiv 13987$ \\
1 & 0 & $13987^2 \equiv 8224$ \\
0 & 0 & $8224^2 \equiv 65529$ \\
\end{tabular}
\end{center}

$$e\cdot3\equiv1\mod \varphi(n)$$

So $d=3$, then we can apply modulo exponentiation to the equation
$$m\equiv c^d\mod n$$
15 & 1 & $1^2 \cdot 3 \equiv 3$ \\
14 & 0 & $3^2 \equiv 9$ \\
13 & 0 & $9^2 \equiv 81$ \\
12 & 0 & $81^2 \equiv 6561$ \\
11 & 0 & $6561^2 \equiv 54449$ \\
10 & 0 & $54449^2 \equiv 61869$ \\
9 & 0 & $61869^2 \equiv 19139$ \\
8 & 0 & $19139^2 \equiv 15028$ \\
7 & 0 & $15028^2 \equiv 282$ \\
6 & 0 & $282^2 \equiv 13987$ \\
5 & 0 & $13987^2 \equiv 8224$ \\
4 & 0 & $8224^2 \equiv 65529$ \\
3 & 0 & $65529^2 \equiv 64$ \\
2 & 0 & $64^2 \equiv 4096$ \\
1 & 0 & $4096^2 \equiv 65281$ \\
0 & 0 & $65281^2 \equiv 65536$ \\


So $m=1415$.

\section{Breaking RSA}

\section{Programming}
In the ex3 folder, with a README file inside it.

\section{Simple Questions}
\begin{enumerate}
\item
\item
\item
\item
\item
$$(97-1)=96=2^5\times3$$
So the generator $x$ should satisfy that 
$$x^{32}\neq 1\mod 97\quad{\rm and}\quad x^{48}\neq 1\mod 97$$
$$x^{16}\neq \pm 1,35,61\mod 97$$
We can find that
\begin{align*}
2^{16}&\equiv61\mod97\\
3^{16}&\equiv61\mod97\\
4^{16}&\equiv1\mod97\\
5^{16}&\equiv36\mod97
\end{align*}
So the smallest generator of $U(Z/97Z)$ is 5.
\end{enumerate}


\end{document}
