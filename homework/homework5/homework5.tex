\documentclass{article}
\usepackage{enumerate}
\usepackage{amsmath}
\usepackage{amssymb}
\usepackage{graphicx}
\usepackage{subfigure}
\usepackage{geometry}
\usepackage{caption}
\usepackage{indentfirst}
\usepackage{multirow}
\geometry{left=3.0cm,right=3.0cm,top=3.0cm,bottom=4.0cm}
\renewcommand{\thesection}{Ex. \arabic{section} ---}
\renewcommand{\mod}{{\rm\ mod\ }}
\title{VE475 Homework 5}
\author{Liu Yihao 515370910207}
\date{}

\begin{document}
\maketitle

\section{RSA setup}
\begin{enumerate}
\item
In the RSA encryption and decryption, we use $$ed\equiv1\mod\varphi(n)$$
$$m^{ed}\equiv m\mod\varphi(n)$$
This is based on the Euler's theorem, which has a condition that $m$ and $n$ be two coprime integers. So it is likely for $n$ to be coprime with $m$.
\item
Suppose $k=a\varphi(n)$, $a\in N^*$, and $m<n$.
\begin{enumerate}[(a)]
\item
\begin{align*}
m^k&\equiv(m^{\varphi(n)})^a\mod n\\
&\equiv1^a\mod n\\
&\equiv1\mod n\\
\end{align*}
So $$m^k\equiv1\mod p\quad{\rm and}\quad m^k\equiv1\mod q$$
\item
First, if $\gcd(m,n)=1$, according to (a), it's obvious that
$$m^{k+1}\equiv m\mod p\quad{\rm and}\quad m^{k+1}\equiv m\mod q$$
Second, if $\gcd(m,n)=p$, so $\gcd(m/p,q)=1$
\begin{align*}
m^{k+1}&\equiv p\left[\left(\frac{m}{p}\right)^{k+1}\mod q\right]\mod n\\
&\equiv p\left[\left(\frac{m}{p}\right)^{a(p-1)\varphi(q)+1}\mod q\right]\mod n\\
&\equiv p\cdot\frac{m}{p}\mod n\\
&\equiv m\mod n
\end{align*}
So $$m^{k+1}\equiv m\mod p\quad{\rm and}\quad m^{k+1}\equiv m\mod q$$
Third, if $\gcd(m,n)=q$, it is similar to the second case.

We can conclude that for any arbitrary $m$, $m^{k+1}\equiv m\mod p{\rm\ and}\mod q$.
\end{enumerate}
\item
\begin{enumerate}[(a)]
\item
We know that $ed\equiv1\mod\varphi(n)$, which means that $ed=k+1$ where $k$ is a multiple of $\varphi(n)$. According to part 2(b), we know that for any arbitrary $m$, $m^{k+1}\equiv m\mod p{\rm\ and}\mod q$, or we can say $m^{k+1}\equiv m\mod n$, so $m^{ed}\equiv m\mod n$,
\item
From the previous calculation, we can find that for all $m<n$, no matter $m$ and $n$ are coprime or not, we can both find that $m^{ed}\equiv m\mod n$, so that the RSA encryption and decryption can be performed. So we can conclude that it is not necessary that $\gcd(m,n)=1$.
\end{enumerate}
\end{enumerate}

\section{RSA decryption}
$$n=11413=101\times113$$

So we can find that $p=101$ and $q=113$, so $\varphi(n)=11200$, and we should calculate $d$ so that $ed\equiv1\mod \varphi(n)$.\\[-0.5em]

By applying the extended euclidean algorithm,
\begin{center}
\begin{tabular}{|c|c|c|c|}
\hline & $q_i$ & $r_i$ & $s_i$ \\\hline
0 & & 7467 & 1 \\\hline
1 & & 11413 & 0 \\\hline
2 & $ 7467\div11413=0$ & $7467-0\times11413=7467$ & $1-0\times0=1$ \\\hline
3 & $ 11413\div7467=1$ & $11413-1\times7467=3946$ & $0-1\times1=-1$ \\\hline
4 & $ 7467\div3946=1$ & $7467-1\times3946=3521$ & $1-1\times-1=2$ \\\hline
5 & $ 3946\div3521=1$ & $3946-1\times3521=425$ & $-1-1\times2=-3$ \\\hline
6 & $ 3521\div425=8$ & $3521-8\times425=121$ & $2-8\times-3=26$ \\\hline
7 & $ 425\div121=3$ & $425-3\times121=62$ & $-3-3\times26=-81$ \\\hline
8 & $ 121\div62=1$ & $121-1\times62=59$ & $26-1\times-81=107$ \\\hline
9 & $ 62\div59=1$ & $62-1\times59=3$ & $-81-1\times107=-188$ \\\hline
10 & $ 59\div3=19$ & $59-19\times3=2$ & $107-19\times-188=3679$ \\\hline
11 & $ 3\div2=1$ & $3-1\times2=1$ & $-188-1\times3679=-3867$ \\\hline
\end{tabular}
\end{center}

$$e\cdot3\equiv1\mod \varphi(n)$$

So $d=3$, then we can apply modulo exponentiation to the equation
$$m\equiv c^d\mod n$$
\begin{center}
\begin{tabular}{c|c|c}
$i$ & $d_i$ & power mod 11413 \\\hline 
12 & 1 & $1^2 \cdot 5859 \equiv 5859$ \\
11 & 1 & $5859^2 \cdot 5859 \equiv 1415$ \\
10 & 1 & $1415^2 \cdot 5859 \equiv 1617$ \\
9 & 0 & $1617^2 \equiv 1112$ \\
8 & 1 & $1112^2 \cdot 5859 \equiv 7374$ \\
7 & 0 & $7374^2 \equiv 4344$ \\
6 & 1 & $4344^2 \cdot 5859 \equiv 6768$ \\
5 & 1 & $6768^2 \cdot 5859 \equiv 4445$ \\
4 & 1 & $4445^2 \cdot 5859 \equiv 4041$ \\
3 & 1 & $4041^2 \cdot 5859 \equiv 11111$ \\
2 & 0 & $11111^2 \equiv 11313$ \\
1 & 1 & $11313^2 \cdot 5859 \equiv 7071$ \\
0 & 0 & $7071^2 \equiv 10101$ \\
\end{tabular}
\end{center}


So $m=1415$.

\section{Breaking RSA}
\begin{enumerate}
\item 
When we decrypt an RSA ciphertext, we use $m\equiv c^d\mod n$. When $d$ is small, the decryption speed will be faster, so one would select short encryption or decryption keys.
\item
$$ed\equiv1\mod{\rm lcm}(p-1,q-1)$$
$$ed=K\cdot{\rm lcm}(p-1,q-1)+1,K\in N$$
Suppose $G=\gcd(p-1,q-1)$, we can find
$$ed=\frac{K}{G}(p-1,q-1)+1$$
Let $k=\dfrac{K}{\gcd(K,G)}$, $g=\dfrac{G}{\gcd(K,G)}$,
$$ed=\frac{k}{g}(p-1,q-1)+1$$
$$\frac{e}{pq}=\frac{k}{dg}(1-\lambda),\lambda=\frac{p+q-1-g/k}{pq}$$
Since $p\approx q\gg0$, $\lambda$ would be very small, then $\dfrac{e}{pq}$ is slightly smaller than $\dfrac{k}{dg}$, and
$$edg=k(p-1)(q-1)+g$$
Let $k_0=\dfrac{k}{g}$ we can find
$$\varphi(n)=(p-1)(q-1)=\frac{ed-1}{k_0}$$
where $\dfrac{k_0}{d}$ converges to $\dfrac{e}{n}$.\\
Then we can apply continued fractions to get a list of approximate of $k_0$ and $d$, validate them and get the right $d$ if it is small enough by the equation
$$x^2-pq+n=0$$
$$x^2-(n-\varphi(n)+1)+n=0$$
$$p,q=\frac{n-\varphi(n)+1\pm\sqrt{(n-\varphi(n)+1)^2-4n}}{2}$$
\item
According to Wiener's theorem, decryption key should be larger than $\dfrac{1}{3}n^{1/4}$. For security considerations, it should be randomly selected from the safe range.
\item
We apply continued fraction to $n$ and $e$ and get the following table:
\begin{center}
\begin{tabular}{c|ccc}
$i$ & $a$ & $k_0$ & $d$ \\\hline
\begin{center}
\begin{tabular}{|c|c|c|c|}
\hline & $q_i$ & $r_i$ & $s_i$ \\\hline
0 & & 7467 & 1 \\\hline
1 & & 11413 & 0 \\\hline
2 & $ 7467\div11413=0$ & $7467-0\times11413=7467$ & $1-0\times0=1$ \\\hline
3 & $ 11413\div7467=1$ & $11413-1\times7467=3946$ & $0-1\times1=-1$ \\\hline
4 & $ 7467\div3946=1$ & $7467-1\times3946=3521$ & $1-1\times-1=2$ \\\hline
5 & $ 3946\div3521=1$ & $3946-1\times3521=425$ & $-1-1\times2=-3$ \\\hline
6 & $ 3521\div425=8$ & $3521-8\times425=121$ & $2-8\times-3=26$ \\\hline
7 & $ 425\div121=3$ & $425-3\times121=62$ & $-3-3\times26=-81$ \\\hline
8 & $ 121\div62=1$ & $121-1\times62=59$ & $26-1\times-81=107$ \\\hline
9 & $ 62\div59=1$ & $62-1\times59=3$ & $-81-1\times107=-188$ \\\hline
10 & $ 59\div3=19$ & $59-19\times3=2$ & $107-19\times-188=3679$ \\\hline
11 & $ 3\div2=1$ & $3-1\times2=1$ & $-188-1\times3679=-3867$ \\\hline
\end{tabular}
\end{center}

\end{tabular}
\end{center}
\end{enumerate}

According to Wiener's theorem, $d<\dfrac{1}{3}n^{1/4}<45$, so we can try data from $i=1,2$.\\

First we can guess that $k_0=1$, $d=4$, $$\phi(n)=\frac{ed-1}{k_0}=310148323$$
$$n-\varphi(n)+1=7791689$$
$$(n-\varphi(n)+1)^2-4n=60709145712677$$

It is not a square number, so $d$ is wrong.\\

Second we can guess that $k_0=9$, $d=37$, $$\phi(n)=\frac{ed-1}{k_0}=\frac{2868871996}{9}$$

It is not a integer, so $d$ is wrong.\\

Third, we can guess that $k_0=10$, $d=41$, $$\phi(n)=\frac{ed-1}{k_0}=317902032$$
$$n-\varphi(n)+1=37980$$
$$(n-\varphi(n)+1)^2-4n=170720356=13066^2$$
$$p=\frac{37980+13066}{2}=25523$$
$$q=\frac{37980-13066}{2}=12457$$
$$n=317940011=25523\times12457$$

\section{Programming}
In the ex3 folder, with a README file inside it.

\section{Simple Questions}
\begin{enumerate}
\item


\item
No, it doesn't. Because the RSA problem is actually a factorization problem. If the attacker succeeded in factoring $n$, no matter how many exponents are chosen, the decryption method is the same.

\item
$$4\cdot516107^2-187722^2\equiv0\mod n$$
$$(2\cdot516107-187722)(2\cdot516107+187722)\equiv0\mod n$$
$$1219936\cdot844492\equiv0\mod n$$
$$64866\cdot844492\equiv0\mod n$$
$$2\cdot3\cdot10811\cdot2^2\cdot211123\equiv0\mod n$$
We can find that 64866 must have a factor of $n$ since $211123<n$ (suppose n have only two factors according to RSA), and the factorization of 10811 is easy since it's small enough. We can try the primes smaller than $\sqrt{10811}(<104)$ and find that it has a factor 19. Then we can deduce that $10811=19\times569$, where 569 is also a prime.\\

At last we can take 3, 19 and 569 as the possible factors of $n$,  validate them and conclude that
$$n=642401=569\times1129$$
\item


\item
$$(97-1)=96=2^5\times3$$
So the generator $x$ should satisfy that 
$$x^{32}\neq 1\mod 97\quad{\rm and}\quad x^{48}\neq 1\mod 97$$
$$x^{16}\neq \pm 1,35,61\mod 97$$
We can find that
\begin{align*}
2^{16}&\equiv61\mod97\\
3^{16}&\equiv61\mod97\\
4^{16}&\equiv1\mod97\\
5^{16}&\equiv36\mod97
\end{align*}
So the smallest generator of $U(Z/97Z)$ is 5.

\item
\begin{enumerate}[(a)]
\item
$$101-1=100=2^2\times5^2$$
\begin{align*}
2^{100/2}&\equiv(2^{10})^5\mod 101\\
&\equiv14^5\mod 101\\
&\equiv100\mod 101\\
2^{100/5}&\equiv(2^{10})^2\mod 101\\
&\equiv14^2\mod 101\\
&\equiv95\mod 101
\end{align*}

Since $2^{50}\not\equiv1\mod101$ and $2^{20}\not\equiv1\mod101$, 2 is a generator of $G$.
\item
$$\log_22=1$$
$$\log_224=\log_23+3\log_22=72$$
\item
$$\log_224=\log_2125=3\log_25=72$$
\end{enumerate}

\item
$$(137-1)=136=2^3\times17$$
\begin{align*}
3^{136/2}&\equiv3^5\cdot(3^{7})^9\mod 137\\
&\equiv243\cdot(-5)^9\mod 137\\
&\equiv106\cdot12^3\mod 137\\
&\equiv106\cdot7\cdot12\mod 137\\
&\equiv136\mod 137\\
3^{136/17}&\equiv3^8\mod 137\\
&\equiv3\cdot-5\mod 137\\
&\equiv122\mod 137
\end{align*}
Since $3^{68}\not\equiv1\mod 137$ and $3^{8}\not\equiv1\mod 137$, 3 is a generator of $U(Z/137Z)$.\\
$$\log_344=6$$
$$\log_32=10$$
$$\log_311=\log_344-2\log_32=-14$$
So $x=122$.
\item
\begin{enumerate}
\item
$$6^5\equiv10\mod11$$
So $6^5=10$ in $U(Z/11Z)$
\item
$$(11-1)=10=2\times5$$
$$2^{10/2}\equiv10\mod 11$$
$$2^{10/5}\equiv4\mod 11$$
Since $2^{5}\not\equiv1\mod11$ and $2^{2}\not\equiv1\mod11$, 2 is a generator of $G$.
\item

\end{enumerate}
\end{enumerate}


\end{document}
