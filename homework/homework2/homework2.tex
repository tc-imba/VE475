\documentclass{article}
\usepackage{enumerate}
\usepackage{amsmath}
\usepackage{amssymb}
\usepackage{graphicx}
\usepackage{subfigure}
\usepackage{geometry}
\usepackage{caption}
\usepackage{indentfirst}
\geometry{left=3.0cm,right=3.0cm,top=3.0cm,bottom=4.0cm}
\renewcommand{\thesection}{Ex. \arabic{section} ---}
\renewcommand{\mod}{{\rm\ mod\ }}
\title{VE475 Homework 1}
\author{Liu Yihao 515370910207}
\date{}

\begin{document}
\maketitle

\section{Simple questions}
\begin{enumerate}
\item \ 
\begin{center}
\begin{tabular}{|c|c|c|c|c|}
\hline
& $q_i$ & $r_i$ & $s_i$ & $t_i$\\\hline
0 & & 17 & 1 & 0\\\hline
1 & & 101 & 0 & 1\\\hline
2 & $17\div101=0$ & $17-0\times101=17$ & $1-0\times0=1$ & $0-0\times1=0$\\\hline
3 & $101\div17=5$ & $101-5\times17=16$ & $0-5\times1=-5$ & $1-5\times0=1$\\\hline
4 & $17\div16=1$ & $17-1\times16=1$ & $1-1\times-5=6$ & $0-1\times1=-1$\\\hline
\end{tabular}
\end{center}
$$17\cdot6\equiv1\mod101$$
So the inverse of 17 modulo 101 is 6.

\item
$$12x\equiv28\mod236$$
$$3x\equiv7\mod59$$
$$3x=59k+7\quad k\in Z$$
First, we can find the solutions in $[0,58]$, try $k=0,1,2$\\
When $k=0$, $x=\dfrac{7}{3}$. When $k=1$, $x=22$. When $k=1$, $x=\dfrac{125}{3}$.\\
So $x=59k+22$, $k\in Z$.

\item
Suppose $m\in[0,30]$ and $c\in[0,30]$, we know
$$c\equiv m^7\mod 31$$
Then we can generate a table from $m$ to $c$.
\begin{center}
\begin{tabular}{cc|cc|cc|cc}
m&c&m&c&m&c&m&c\\\hline
 0 & 0 & 1 & 1 & 2 & 4 & 3 & 17 \\
 4 & 16 & 5 & 5 & 6 & 6 & 7 & 28 \\
 8 & 2 & 9 & 10 & 10 & 20 & 11 & 13 \\
 12 & 24 & 13 & 22 & 14 & 19 & 15 & 23 \\
 16 & 8 & 17 & 12 & 18 & 9 & 19 & 7 \\
 20 & 18 & 21 & 11 & 22 & 21 & 23 & 29 \\
 24 & 3 & 25 & 25 & 26 & 26 & 27 & 15 \\
 28 & 14 & 29 & 27 & 30 & 30 &
\end{tabular}
\end{center}
Obviously there is a bijection between the plaintext m and the ciphertext c, and we can simply decrypt the message according to the table above.

\item
$$\sqrt{4369}<\sqrt{4883}<70$$
Consider all of the primes in $[2,70]$, they are 2, 3, 5, 7, 11, 13, 17, 19, 23, 29, 31, 37, 41, 43, 47, 53, 59, 61, 67.\\
For 4883, first, try to divide 4883 by them one by one, we can find that $4883=19\times 257$. Then, try to divide 257 by 2, 3, 5, 7, 11, 13, all of them have a reminder, so 257 is a prime, $4883=19\times 257$.\\
For 4369, it's interesting because $4883=4369+2\times257$, so $4369=17\times257$, where 17 and 257 are primes.

\item
$$A=\begin{pmatrix}3&5\\7&3\\\end{pmatrix}\mod p$$
When $p=2$, 
$$A=\begin{pmatrix}1&1\\1&1\\\end{pmatrix}\mod 2$$
$$\det\begin{pmatrix}1&1\\1&1\\\end{pmatrix}=0$$
It is not invertible.\\

When $p=3$,
$$A=\begin{pmatrix}0&2\\1&0\\\end{pmatrix}\mod 3$$
$$\det\begin{pmatrix}0&2\\1&0\\\end{pmatrix}=-2$$
It is invertible.\\

When $p=5$,
$$A=\begin{pmatrix}3&0\\2&3\\\end{pmatrix}\mod 5$$
$$\det\begin{pmatrix}3&0\\2&3\\\end{pmatrix}=9$$
It is invertible.\\

When $p=7$,
$$A=\begin{pmatrix}3&5\\0&3\\\end{pmatrix}\mod 7$$
$$\det\begin{pmatrix}3&5\\0&3\\\end{pmatrix}=9$$
It is invertible.\\

When $p>7$,
$$A=\begin{pmatrix}3&5\\7&3\\\end{pmatrix}\mod 7$$
$$\det\begin{pmatrix}3&5\\7&3\\\end{pmatrix}=-26$$
It is invertible.\\

So when $p=2$, it is not invertible.

\item
\begin{align*}
2^{2017}&\equiv 2\cdot 4^{1008}\mod 5\\
&\equiv 2\cdot (-1)^{1008}\mod 5\\
&\equiv 2\mod 5\\
2^{2017}&\equiv 2\cdot 64^{336}\mod 13\\
&\equiv 2\cdot (-1)^{336}\mod 13\\
&\equiv 2\mod 13\\
2^{2017}&\equiv 4\cdot 32^{403}\mod 31\\
&\equiv 4\cdot 1^{403}\mod 31\\
&\equiv 4\mod 31
\end{align*}

Then we can apply the Chinese remainder theorem to solve $2^{2017}$ modulo 2015.

\begin{align*}
5\cdot13&\equiv 3\mod 31\\
65\cdot-10&\equiv 1\mod 31\\
13\cdot31&\equiv 3\mod 5\\
403\cdot2&\equiv 1\mod 5\\
5\cdot31&\equiv -1\mod 13\\
155\cdot-1&\equiv 1\mod 13
\end{align*}
\begin{align*}
2^{2017}&\equiv -650\cdot4+806\cdot2-155\cdot2\mod 2015\\
&\equiv -1298\mod 2015\\
&\equiv 717\mod 2015\\
\end{align*}

\item
According to Assignment 1/Ex. 1/3, let $a$, $b$ and $n$ be three positive integers such that $n\mid ab$ and $\gcd(a,n)=1$, we can prove that $n\mid b$.\\
Now let $n=p$, since $p$ is a prime, we know $\gcd(a,p)=1$ or $\gcd(a,p)=p$. If $\gcd(a,p)=1$, according to the conclusion above, $p\mid b$, so $b\equiv0\mod p$. If $\gcd(a,p)=p$, we can simply get $a\equiv0\mod p$, so it is proved.

\end{enumerate}

\section{Rabin cryptosystem}
\begin{enumerate}
\item
Rabin cryptosystem is an asymmetric cryptosystem, which uses both a public key and a private key. The public key is necessary for later encryption and can be published, while the private key must be possessed only by the recipient of the message.

The keys can be generated by this method: choose two large distinct primes $p$ and $q$ as the private keys, and $n=p\cdot q$ as the public key.

For the encryption, suppose the plaintext $m\in[0,n-1]$, the ciphertext $c$ is determined by $$c=m^2\mod n$$

For the decryption, the private keys are necessary. We can find the plaintext $m$ by $$m^2\equiv c\mod m$$

There is no efficient method known for the finding of $m$ if we don't have the private keys. However, if we know $p$ and $q$, the Chinese remainder theorem can be applied to solve for $m$.

$$m_p=\sqrt{c}\mod p$$
$$m_q=\sqrt{c}\mod q$$

By applying the the extended Euclidean algorithm, we wish to find $y_{p}$ and $y_{q}$ such that $y_p\cdot p+y_q\cdot q=1$.

\item


\end{enumerate}


\end{document}
