\documentclass{article}
\usepackage{enumerate}
\usepackage{amsmath}
\usepackage{amssymb}
\usepackage{graphicx}
\usepackage{subfigure}
\usepackage{geometry}
\usepackage{caption}
\usepackage{indentfirst}
\usepackage{multirow}
\geometry{left=3.0cm,right=3.0cm,top=3.0cm,bottom=4.0cm}
\renewcommand{\thesection}{Ex. \arabic{section} ---}
\renewcommand{\mod}{{\rm\ mod\ }}
\title{VE475 Homework 7}
\author{Liu Yihao 515370910207}
\date{}

\begin{document}
\maketitle

\section{Cramer-Shoup cryptosystem}
\begin{enumerate}
\item
Cramer–Shoup cryptosystem consists of three algorithms: the key generator, the encryption algorithm, and the decryption algorithm.
\begin{enumerate}[a)]
\item The key generator \\
First, Alice generates a cyclic group $G$ of order $q$ and finds two generators $g_1$ and $g_2$ for it. Then she randomly chooses $x_1,x_2,y_1,y_2,z$ from $\{0,\dots,q-1\}$ and computes $c=g_1^{x_1}g_2^{x_2}$, $d=g_1^{y_1}g_2^{y_2}$ and $h=g_1^z$. At last, she publishes $(c,d,h,G,q,g_1,g_2)$ as the public key and keeps $(x_1,x_2,y_1,y_2,z)$ as the private key.
\item The encryption algorithm \\
First, Bob converts $m$ into an element of $G$ and choose a random $k$ from $\{0,\dots,q-1\}$. Then he computes $u_1=g_1^k$, $u_2=g_2^k$, $e=h^km$, $\alpha=H(u_1,u_2,e)$ where $H(x)$ is a collision-resistant cryptographic hash function, and $v=c^kd^{k\alpha}$ At last, he sends the ciphertext $(u_1,u_2,e,v)$ to Alice.
\item The decryption algorithm \\
First, Alice computes $\alpha=H(u_1,u_2,e)$ and verifies that $u_1^{x_1}u_2^{x_2}(u_1^{y_1}u_2^{y_2})^\alpha=v$. If the verification fails, the decryption algorithm ends with failure output. Otherwise, she computes the plaintext $m=e/h^k$. The decryption stage correctly decrypts any properly-formed ciphertext, since $u_1^z=g_1^{kz}=h^k$.
\end{enumerate}

\item
Adaptive chosen ciphertext attacks can be applied if a ciphertext can be modified in specific ways that will have a predictable effect on the decryption of that message. However, The decryption algorithm of Cramer-Shoup cryptosystem rejects all invalid ciphertexts constructed by an attacker through verifying the result generated by a collision-resistant cryptographic hash function. It limits ciphertext malleability so that it can be considered secure under this kind of attack.

\item
\begin{enumerate}[a)]
\item Similarities: Both are public key cryptosystems computed in a cyclic group $G$, the private keys are both based on the difficulty of solving Discrete Logarithm Problem.
\item Differences: Cramer–Shoup cryptosystem consists a collision-resistant cryptographic hash function which is used to verify the ciphertext while Elgamal cryptosystem doesn't.

\end{enumerate}

\end{enumerate}


\section{Simple questions}
\begin{enumerate}
\item
Since $p$ is a prime and $p\nmid \alpha$, we can find $\gcd(p,\alpha)=1$, so $\alpha^{p-1}\equiv 1\mod p$. First, $h(x)$ isn't second pre-image resistant. Given $x$, we can simply find $x'=x+p-1$ so that $h(x)=h(x')$. Second, $h(x)$ isn't collision resistant. For any $x$, we can simply find $x'=x+p-1$ so that $h(x)=h(x')$. So it is not a good cryptographic hash function.
\item
\begin{align*}
\lfloor2^{30}\sqrt{2}\rfloor&=\lfloor40000000\cdot\sqrt{2}\rfloor=5A827999 \\
\lfloor2^{30}\sqrt{3}\rfloor&=\lfloor40000000\cdot\sqrt{3}\rfloor=6ED9EBA1 \\
\lfloor2^{30}\sqrt{5}\rfloor&=\lfloor40000000\cdot\sqrt{5}\rfloor=8F1BBCDC \\
\lfloor2^{30}\sqrt{10}\rfloor&=\lfloor40000000\cdot\sqrt{10}\rfloor=CA62C1D6
\end{align*}
I found the results identical to $K_0||\cdots||K_{19}$, $K_{20}||\cdots||K_{39}$, $K_{40}||\cdots||K_{59}$ and $K_{60}||\cdots||K_{79}$.
\end{enumerate}


\section{Birthday paradox}
\begin{enumerate}
\item
Since $g(x)=\ln(1-x)+x+x^2$, we know
$$g'(x)=-\frac{1}{1-x}+1+2x$$
When $g'(x)=0$,
$$1+x-1+2x(x-1)=0$$
$$x_1=0,x_2=\frac{1}{2}$$
$$g''(x)=-\frac{1}{(x-1)^2}+2$$
$$g(0)=1,{\rm\ it\ is\ a\ local\ minimum\ point}$$
$$g\left(\frac{1}{2}\right)=-2,{\rm\ it\ is\ a\ local\ maximum\ point}$$
So we can conclude that when $x\in\left[0,\dfrac{1}{2}\right]$, $g(x)\in\left[g(0),g\left(\dfrac{1}{2}\right)\right]\geqslant 0$

Similarly, let $h(x)=\ln(1-x)+x$, we know
$$h'(x)=-\frac{1}{1-x}+1$$
When $h'(x)=0$,
$$1+x-1=0$$
$$x=0$$
$$h''(x)=-\frac{1}{(x-1)^2}$$
$$h(0)=-1,{\rm\ it\ is\ a\ local\ maximum\ point}$$
So we can conclude that when $x\in\left[0,\dfrac{1}{2}\right]$, $h(x)\in\left[h\left(\dfrac{1}{2}\right),h(0)\right]\leqslant 0$

According to the above, $$-x-x^2 \leqslant \ln(1-x) \leqslant -x$$

\item
Since $j\in[1,r-1]$ and $r\leqslant \dfrac{n}{2}$, we can find that $\dfrac{j}{n}\in\left[0,\dfrac{1}{2}\right]$, so
$$-\frac{j}{n}-\left(\frac{j}{n}\right)^2 \leqslant \ln\left(1-\frac{j}{n}\right) \leqslant -\frac{j}{n}$$
$$\sum_{j=1}^{r-1} \left[ -\frac{j}{n}-\left(\frac{j}{n}\right)^2 \right] \leqslant \sum_{j=1}^{r-1} \ln\left(1-\frac{j}{n}\right) \leqslant \sum_{j=1}^{r-1} -\frac{j}{n}$$
$$-\frac{(r-1)r}{2n}-\frac{(r-1)r(2r-1)}{6n^2} \leqslant \sum_{j=1}^{r-1} \ln\left(1-\frac{j}{n}\right) \leqslant -\frac{(r-1)r}{2n}$$

When $r>1$, $$\frac{(r-1)r(2r-1)}{6n^2}=\frac{r^3-\frac{3}{2}r^2+r}{3n^2}<\frac{r^3}{3n^2}$$
$$-\frac{(r-1)r}{2n}-\frac{r^3}{3n^2} \leqslant \sum_{j=1}^{r-1} \ln\left(1-\frac{j}{n}\right) \leqslant -\frac{(r-1)r}{2n}$$

\item
Exponentiate the inequation above, we can get
$$\exp\left(-\frac{(r-1)r}{2n}-\frac{r^3}{3n^2}\right) \leqslant \prod_{j=1}^{r-1} \left(1-\frac{j}{n}\right) \leqslant \exp\left(-\frac{(r-1)r}{2n}\right)$$

Let $\lambda=\dfrac{r^2}{2n}$, $c_1=\sqrt{\dfrac{\lambda}{2}}-\dfrac{(2\lambda)^{3/2}}{3}$ and $c_2=\sqrt{\dfrac{\lambda}{2}}$.
$$-\lambda+\frac{c_1}{\sqrt{n}}=-\frac{r^2}{2n}+\frac{r}{2n}-\frac{r^3}{n^2}=-\frac{(r-1)r}{2n}-\frac{r^3}{3n^2}$$
$$-\lambda+\frac{c_2}{\sqrt{n}}=-\frac{r^2}{2n}+\frac{r}{2n}=-\frac{(r-1)r}{2n}$$

So $$e^{-\lambda}e^{c_1/\sqrt{n}} \leqslant \prod_{j=1}^{r-1} \left(1-\frac{j}{n}\right) \leqslant e^{-\lambda}e^{c_2/\sqrt{n}}$$

\item
If $n$ is large and $\lambda<\dfrac{n}{8}$
$$\lambda=\frac{r^2}{2n}<\frac{n}{8}$$
$$r<\frac{n}{2}$$

Since $\lambda$ is a constant, $c_1$ and $c_2$ are also constants.
$$\lim_{n\to\infty}e^{c_1/\sqrt{n}}=\lim_{n\to\infty}e^0=1$$
$$\lim_{n\to\infty}e^{c_2/\sqrt{n}}=\lim_{n\to\infty}e^0=1$$

Then we can conclude that
$$\prod_{j=1}^{r-1} \left(1-\frac{j}{n}\right) \approx e^{-\lambda}$$
\end{enumerate}

\section{Birthday attack}
\begin{enumerate}
\item
$$P=1-\prod_{j=1}^{39}\left(1-\frac{j}{1000}\right)\approx0.5464$$
\item
$$P=39\left(\frac{1}{1000}\right)\left(\frac{999}{1000}\right)^{38}\approx0.0375$$
\item

\end{enumerate}

\section{Faster multiple modular exponentiation}
\begin{enumerate}
\item
The complexity of computing $\alpha^a\mod n$ is $O(\log a)$, the complexity of computing $\beta^b\mod n$ is $O(\log b)$, so the total time complexity is $O(\log ab)$.
\item

\item
\item
\end{enumerate}

\end{document}
