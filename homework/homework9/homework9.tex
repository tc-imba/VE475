\documentclass{article}
\usepackage{enumerate}
\usepackage{amsmath}
\usepackage{amssymb}
\usepackage{graphicx}
\usepackage{subfigure}
\usepackage{geometry}
\usepackage{caption}
\usepackage{indentfirst}
\usepackage{multirow}
\usepackage{algorithm}  
\usepackage{algorithmicx}  
\usepackage{algpseudocode}
\geometry{left=3.0cm,right=3.0cm,top=3.0cm,bottom=4.0cm}
\renewcommand{\thesection}{Ex. \arabic{section} ---}
\renewcommand{\mod}{{\rm\ mod\ }}
\title{VE475 Homework 9}
\author{Liu Yihao 515370910207}
\date{}

\begin{document}
\maketitle

\section{Missile or not missile}

$(t, w)$-threshold scheme can be used. Give each desk clerk 2 share, each colonel 5 shares, and the general 10 shares. Let $t=10$ and $w=30$, the problem is solved.

\section{Asmuth-Bloom Threshold Secret Sharing Scheme}
In order to take advantage of the Chinese remainder Theorem, the Asmuth-Bloom Threshold Secret Sharing Scheme can be applied.

The setting up procedure is: Let $2 \leqslant k \leqslant n$ be integers. We consider a sequence of pairwise coprime positive integers $m_0<m_1<\cdots<m_n$ such that $m_0\cdot m_{n-k+2}\cdots m_n<m_1\cdots m_k$. For this given sequence, we choose the secret $S$ as a random integer in the set $Z/m_0Z$.

Then we can generate a random integer $\alpha$ so that $S+\alpha\cdot m_0<m_1\cdots m_k$. After computing $s_i \equiv S+\alpha m_0 \mod m_i$ for $1\leqslant i\leqslant n$, we can get the shares $I_i=\langle s_i,m_i\rangle$. Now we can take any of $k$ different shares from $n$ shares, $I_{i_1,}I_{i_2},\dots,I_{i_k}$, so that

$$
\left\{\begin{aligned}
x &\equiv s_{i_1} \mod m_{i_1} \\
x &\equiv s_{i_2} \mod m_{i_2} \\
& \vdots \\
x &\equiv s_{i_k} \mod m_{i_k} \\
\end{aligned}\right.
$$

According to the Chinese remainder Theorem, we can decide a unique $x<m_{i_1}\cdot m_{i_2}\cdots m_{i_k}$. By the construction of the shares, we can get $$S \equiv x \mod m_0$$

So a Threshold Secret Sharing Scheme is built.

\section{Shamir's Threshold Secret Sharing Scheme}

The Lagrange’s interpolation method is

$$L_i(x)=\frac{(x-x_0)\cdots(x-x_{i-1})(x-x_{i+1})\cdots(x-x_n)}{(x_i-x_0)\cdots(x_i-x_{i-1})(x_i-x_{i+1})\cdots(x_i-x_n)}$$
$$$$
$$p(x)=\sum_{i=0}^n y_iL_i(x)$$

When it is applied to the scheme, all of the values should modulo $p$, so a $k$ multiple of $$\frac{p(x-x_0)\cdots(x-x_{i-1})(x-x_{i+1})\cdots(x-x_n)}{(x_i-x_0)\cdots(x_i-x_{i-1})(x_i-x_{i+1})\cdots(x_i-x_n)}$$ 

should be added to each $L_i(x)$ when reconstructing the polynomial $p(x)$ in order to ensure each parameter is integer. \\

Similar to the lecture, let $p=1234567890133$, $m=190503180520$, $r_1=482943028839$, $r_2=1206749628665$, we can choose 2, 3 and 7, so that
\begin{align*}
x_0=2 &\quad y_0=1045116192326 \\
x_1=3 &\quad y_1=154400023692 \\
x_2=7 &\quad y_2=973441680328
\end{align*}

Then we can get
\begin{align*}
L_0(x) &= \frac{(x-3)(x-7)}{(2-3)(2-7)}=\frac{1}{5}(x-3)(x-7) \\
L_1(x) &= \frac{(x-2)(x-7)}{(3-2)(3-7)}=-\frac{1}{4}(x-2)(x-7) \\
L_2(x) &= \frac{(x-2)(x-3)}{(7-2)(7-3)}=\frac{1}{20}(x-2)(x-3)
\end{align*}

\begin{align*}
p(x) &= y_0L_0(x)+y_1L_1(x)+y_2L_2(x) \\
&= \frac{1045116192326}{5}(x-3)(x-7)-\frac{154400023692}{4}(x-2)(x-7)+\frac{973441680328}{20}(x-2)(x-3) \\
&= \frac{1095476582793}{5}x^2-1986192751427x+\frac{20705602144728}{5}
\end{align*}
$$r_2 \equiv \frac{1095476582793}{5}+\frac{4}{5}p \equiv 1206749628665 \mod p$$
$$r_1 \equiv -1986192751427 \equiv 482943028839 \mod p$$
$$m \equiv \frac{20705602144728}{5}+\frac{4}{5}p \equiv 190503180520 \mod p$$

\section{Simple questions}
\begin{enumerate}
\item
$$z=2x+3y+13=5x+3y+1$$
$$x=4,z=3y+21$$
So the secret value $x$ is 4.
\item
First, for $n=2$, $$\det V_2=x_2-x_1=\prod_{1\leqslant j\leqslant k\leqslant 2}(x_k-x_j)$$
Second, for $n=m\geqslant 2$, suppose $$\det V_m=\prod_{1\leqslant j\leqslant k\leqslant m}(x_k-x_j)$$
When $n=m+1$, 
$$\det V_{m+1}=
\begin{vmatrix}
1 & x_1 & \cdots & x_1^{m-1} & x_1^m \\
1 & x_2 & \cdots & x_2^{m-1} & x_2^m \\
\vdots & \vdots & \ddots & \vdots & \vdots \\
1 & x_m & \cdots & x_m^{m-1} & x_m^m \\
1 & x_{m+1}^2 & \cdots & x_{m+1}^{m-1} & x_{m+1}^m \\
\end{vmatrix}
$$
From the last column to the second column, multiply the left column by $-x_{m+1}$ and add it to that column, we can get
$$
\begin{aligned}
\det V_{m+1}&=
\begin{vmatrix}
1 & x_1-x_{m+1} & \cdots & x_1^{m-2}(x_1-x_{m+1}) & x_1^{m-1}(x_1-x_{m+1}) \\
1 & x_2-x_{m+1} & \cdots & x_2^{m-2}(x_2-x_{m+1}) & x_2^{m-1}(x_2-x_{m+1}) \\
\vdots & \vdots & \ddots & \vdots & \vdots \\
1 & x_m-x_{m+1} & \cdots & x_m^{m-2}(x_m-x_{m+1}) & x_m^{m-1}(x_2-x_{m+1}) \\
1 & 0 & \cdots & 0 & 0 \\
\end{vmatrix}\\
&=\prod_{i=1}^m(x_{m+1}-x_i)
\begin{vmatrix}
1 & x_1 & \cdots & x_1^{m-2} & x_1^{m-1} \\
1 & x_2 & \cdots & x_2^{m-2} & x_2^{m-1} \\
\vdots & \vdots & \ddots & \vdots & \vdots \\
1 & x_{m-1} & \cdots & x_{m-1}^{m-2} & x_{m-1}^{m-1} \\
1 & x_m^2 & \cdots & x_m^{m-2} & x_m^{m-1} \\
\end{vmatrix}\\
&=\prod_{i=1}^m(x_{m+1}-x_i)\det V_m\\
&=\prod_{i=1}^m(x_{m+1}-x_i)\prod_{1\leqslant j\leqslant k\leqslant m}(x_k-x_j)\\
&=\prod_{1\leqslant j\leqslant k\leqslant m+1}(x_k-x_j)
\end{aligned}
$$
So it is proved.
\end{enumerate}

\section{Reed Solomon codes}
\begin{enumerate}
\item
The Reed Solomon code have three parameters: an alphabet size $q$, a block length $n$ and a message length $k$ with $k<n\leqslant q$. $q$ has to be a prime power so that a finite field of order q can be formed. The block length is usually some constant multiple of the message length and is equal to or one less than the alphabet size, that is, $n=q$ or $n=q-1$.

Every codeword of the Reed Solomon code is a sequence of function values of a polynomial $p$ of degree less than $k$. The message is interpreted as the description of a polynomial $p$ of degree less than $k$ over the finite field $F$ with $q$ elements. In turn, the polynomial p is evaluated at n distinct points $a_1,a_2,\dots,a_n$ of the field F, and the sequence of values is the corresponding codeword. The set $\mathcal{C}$ of codewords of the Reed Solomon code is defined as follows:

$$\mathcal{C}=\{(p(a_1),p(a_1),\dots,p(a_3)\}$$

\item
Since any two distinct polynomials of degree less than $k$ agree in at most $k-1$ points, this means that any two codewords of the Reed Solomon code disagree in at least $n-k+1$ positions. So the distance $D$ of the Reed-Solomon code is $n-k+1$.

According to Theorem 7.16, we know it is possible to identify a parent of a descendant of $\mathcal{C}\subset(F_q)^n$ if
$$D>n(1-\frac{1}{w^2})$$
When $w=2$,
$$n-k+1>\frac{3}{4}n$$
So
$$n>4k-4$$
\end{enumerate}

\end{document}
